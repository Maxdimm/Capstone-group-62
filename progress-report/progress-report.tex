\documentclass[letterpaper,10pt,draftclsnofoot,onecolumn,titlepage]{IEEEtran}

\usepackage{graphicx}
\usepackage{amssymb}
\usepackage{amsmath}
\usepackage{amsthm}
\usepackage{alltt}
\usepackage{float}
\usepackage{color}
\usepackage{url}
\usepackage{enumitem}
\usepackage{pstricks, pst-node}
\usepackage{geometry}
\usepackage{array}


\geometry{margin = .75in}

\usepackage{hyperref}



\newcommand*{\signature}[1]{%
	\par\noindent\makebox[3.5in]{\hrulefill} \hfill\makebox[3.0in]{\hrulefill}%
	\par\noindent\makebox[3.5in][l]{#1}	    \hfill\makebox[3.0in][l]{Date}%
}%

\def\name{Kevin Stine, Courtney Bonn, Maxwell Dimm}
\def\team{Calvary Chapel Corvallis}
\def\grp{Group \#62}

\hypersetup{
	colorlinks = true,
	urlcolor = black,
	linkcolor = black,
	pdfauthor = {\name},
	pdftitle = {CS461 Design Document},
	pdfsubject = {CS461 Design Document},
	pdfpagemode = UseNone
}

\begin{document}
	\title{\huge \team \\ Progress Report\\ CS 461 Fall 2016}
	\author{\large \name \\ \grp}



	\maketitle

		\begin{abstract}The purpose of this project is to produce an iOS/Android application for Calvary Chapel of Corvallis that will allow members to access a plethora of information all in one localized space.
		The Church's current website does not provide an interface where current members of the church can very quickly access important information such as events, bulletins, and messages from the service.
		The desired application will be simple enough for anyone to use while providing back end access for staff to easily upload new information to the app.
		The priorities lie in maximizing the usability of the app and providing bulletin, schedule, video, and giving functionality.
		We will work with the existing Calvary Chapel web development team to create a product that is seamlessly integrated with their already existing network.
		\end{abstract}

		\clearpage

		\section{Purpose and Goals}
		The purpose of our project is to create an application for Calvary Corvallis Church that will act as a connection between the congregation and the administration.
		The church already has a website that has some of this information, but they want the website and app to serve different functions. 
		The website will be to introduce people to the church.
		The app will be used for the existing congregation as the go to place to access the most commonly used or needed information.
		Some of the features being provided within the app are: having sermons available, listing the bulletin, having the church schedule, and allowing members to donate to the church.
		Our client explained to us that these were the features that they wanted in the app as they are the most needed services by their members.

		Our client has requested that the app be as automated as possible in regards to updating the information hosted within it as to reduce any upkeep as much as possible.
		So we will be working with their existing infrastructure as much as possible to pull our information from.
		We are also creating both an iOS and Android application and we want to make the applications as functionally similar as possible.
		This will allow for greater understanding of the app across users who may or may not be super tech savvy.
		Our final goal in this project is to reduce costs wherever possible for our client.
		If that means suggesting newer cheaper infrastructure or setting up our app in a way that reduces how often it will need to be updated, we want to do it.

		\section{Progress report}
		Currently, we are still in the planning phase of the project. 
		We have just finished the design document which detailed how we plan on implementing our app and how it will be designed. 
		However, we have not begun the actual implementation of either app quite yet. 
		Our immediate plans include extensive research on iOS and Android app development which we will focus on during Winter break. 
		

		\section{Problems}



		\section{Reflection}
			\begin{table}[H]
			\caption{Retrospective on Fall 2016}
			\begin{center}
				\begin{tabular}{| p{0.06\linewidth} | p{0.28\linewidth} | p{0.28\linewidth} | p{0.28\linewidth} | }
					\hline
					\textbf{Time} & \textbf{Positives} & \textbf{Deltas} & \textbf{Actions} \\ [0.5ex]
					%heading
					\hline
					Week 3 & We met with our client for the first time in week 3. We found out a more detailed idea of what they wanted from us. Also we got to meet 2/3 of their development team.  & No changes needed as of week 3 & We needed to become more familiar with working with LaTeX. \\
					\hline
					Week 4 & We had a second meeting with our client. We met the last developer on their team. He had minimal app experience but had an apple developer account. & No changes needed as of week 4 & We need to start thinking about problem statement document. \\
					\hline
					Week 5 & We met our TA, Vee, for the first time. We also finished up our problem statement. & No changes needed as of week 5. & We need to figure out if we need to develop one or two apps for the different platforms. \\
					\hline
					Week 6 & We focused on finishing our requirements document and planning a timeline for the rest of the project. & Our client wanted us to change the app name to its official name in the requirements document. & We need to start researching our technology review and begin the design document. \\
					\hline
					Week 7 & We began working on our technology review and splitting up the parts of our system. & No changes as of week 7 & We need to continue researching the different technologies and keep researching our respective parts. \\
					\hline
					Week 8 & We finished our technology review and began looking into the design document. & No changes as of week 8 & We have to put a lot of effort in our design document in order to get it to our client on time. \\
					\hline
					Week 9 & We got a headstart on the design document, but didn't get as far because of the Holiday. & No changes as of week 9 & With the term coming to an end, we have to finish up the design document and send it to our client. \\
					\hline
					Week 10 & . & . & . \\
					\hline

				\end{tabular}
			\end{center}
			\end{table}

\end{document}
