\documentclass[letterpaper,10pt,draftclsnofoot,onecolumn,titlepage]{IEEEtran}

\usepackage{graphicx}
\usepackage{amssymb}
\usepackage{amsmath}
\usepackage{amsthm}
\usepackage{alltt}
\usepackage{float}
\usepackage{color}
\usepackage{url}
\usepackage{enumitem}
\usepackage{pstricks, pst-node}
\usepackage{geometry}
\usepackage{array}


\geometry{margin = .75in}

\usepackage{hyperref}

\newcommand*{\signature}[1]{%
	\par\noindent\makebox[3.5in]{\hrulefill} \hfill\makebox[3.0in]{\hrulefill}%
	\par\noindent\makebox[3.5in][l]{#1}	    \hfill\makebox[3.0in][l]{Date}%
}%

\def\name{Kevin Stine, Courtney Bonn, Maxwell Dimm}
\def\team{Calvary Chapel Corvallis}

\hypersetup{
	colorlinks = true,
	urlcolor = black,
	linkcolor = black,
	pdfauthor = {\name},
	pdftitle = {CS461 Requirements},
	pdfsubject = {CS461 Requirements},
	pdfpagemode = UseNone
}

\begin{document}
	\title{\huge \team \\ Technology Review \\ CS 461 Fall 2016}
	\author{\large \name}

	\maketitle
		\begin{abstract}The purpose of this project is to produce an iOS/Android application for Calvary Chapel of Corvallis that will allow members to access a plethora of information all in one localized space.
		The Church's current website doesn't provide an interface where current members of the church can very quickly access important information such as events, bulletins, and messages from the service.
		The desired application will be simple enough for anyone to use while providing back end access for staff to easily upload new information to the app.
		The priorities lie in maximizing the usability of the app and providing bulletin, schedule, video, and giving functionality.
		We will work with the existing Calvary Chapel web development team to create a product that is seamlessly integrated with their already existing network.
		\end{abstract}
		
	
	\vspace{2in}
	\signature{Client}
	\vspace{.25in}
	\signature{Courtney Bonn}
	\vspace{.25in}
	\signature{Kevin Stine}
	\vspace{.25in}
	\signature{Maxwell Dimm}

	\clearpage

	\tableofcontents

	\clearpage

	\section{Introduction}
	This document will break down our application in 9 different parts.
	There will be an outline on 3 different types of technology that can be used to implement the 9 different parts of the system. 
	We will detail each type of technology as well as discuss why it could be useful for the project. 
	We will then compare each type of technology against each other and analyze which one would be the best for us to use in production. 
	The 9 pieces of the app are divided among the three group members. 
	iOS development platform, iOS user interface organization, and integrating the e-bulletins from the current website will be researched and handled by Courtney Bonn.
	Android development platform, Android user interface organization, and integrating the current giving platform will be researched and handled by Kevin Stine. 
	Finally, cross-platform development,  integrating the recent sermons, and integrating the existing calendar will be researched and handled by Max Dimm. 
	
	 
	\section{iOS Development Platform}
	\subsection{Xcode using Swift}
	The main technology used to develop iOS applications is Xcode. 
	Xcode is a program that can be downloaded on any Mac OS platform and isn't able to be downloaded on Windows PCs.
	It makes use of developer tools to allow users to create an iOS app that will work with all updated iOS devices. 
	This tool uses the language of Swift, which is a new programming language that has been released by Apple \cite{AppleSwift}. 
	For people who have never developed an app before, Swift is considered the easier language to learn \cite{CodeChris}. 
	The tool allows the developer to use a simulator that will display the graphical interface that the user will see and makes the process of developing an app for iOS devices go smoothly. 
	
	\subsection{Xcode using Objective-C}
	A different toolkit to use when developing iOS applications, is similar to the previous option and still takes advantage of the Xcode program, but instead uses the language Objective-C \cite{AppleObjC}. 
	In order to build an app that will work with iOS 7, Objective-C is the language one would need to use. 
	However, while there are some people who may want to use our app that still have a phone with iOS 7, the majority of users will more than likely be updated to iOS 9 or iOS 10. 
	Not only is Swift recommended for newer versions of iOS devices, it's also suggested that Objective-C is a more difficult language to learn with little experience of app development \cite{CodeChris}. 
	\subsection{iPhone App Builder}
	The third option for a toolkit to build an iOS app is a program that will essentially build the app for you, leaving no need to learn how to code. 
	One program, AppyPie \cite{AppyPie}, boasts that it will take all of the work out of creating an app and allow an unskilled user to get a working app quickly uploading to the App store.
	This option would be useful for people who have no prior programming experience and are just looking to get an app produced with little effort. 
	\subsection{Goals}
	The goal for comparing iOS development tool is to acquire which way is most acceptable and makes most sense for developing an iOS application. 
	We want to determine which option is relevant to current applications and which option will produce a valid application. 
	\subsection{Criteria Evaluation}
	The criteria used for each option is relevancy to current application development, ease of which one can learn how to use the programming language, and the level of difficulty one is looking for when developing the app.
	\subsection{Table Comparison}
	
	\begin{table}[ht]
	\caption{Comparison of iOS Development Platforms}
	\begin{center}
	\begin{tabular} { | m{3cm} | m{5cm} | m{5cm} | }
	\hline\hline
	Tool & Brief description & Rationale \\ [0.5ex]
	%heading
	\hline
	Xcode using Swift & Using the Swift programming language to write the application using the program Xcode & Best option as it's the most recent and will allow us to actually learn how to develop an app \\
	\hline
	Xcode using Objective-C & Using the Objective-C programming language to write the application using the program Xcode & Would be a good option if we needed an iOS 7 app, but not if we want an app that supports recent iOS versions \\
	\hline
	iPhone App Builder & Using a third-party program to automatically build the app & A good option for people who have no programming experience and just want to quickly produce an app \\
	\hline
	\end{tabular}
	\end{center}
	\end{table}
	
	\subsection{Discussion}
	Using Xcode with Swift is a strong option for building iOS--some would say that it is one of the only options.
	In order for an app to be backwards compatible with iOS 7, the app would need to be programmed with Objective-C. 
	But noting the fact that iOS 7 is 3 years old and very few (if any) iOS devices are still running this software version, it's somewhat safe to say that having an app that can be downloaded on iOS 7 is of little concern. 
	
	The idea of using a third-party program to build the app automatically is tempting, but would defeat the purpose of learning how to develop an iOS app. 
	If the objective of this project was to produce an iOS app within a matter of days or weeks, then this option could be considered useful. 
	\subsection{Selection}
	Because our main objective for this project is to learn the detail of producing an iOS app, it is best to choose Xcode with Swift. 
	Swift is the most updated language for iOS development and using Xcode simplifies the process of building the application. 
	With Xcode and Swift, we will be able to properly produce an application that can be submitted to the App Store for download. 
	It will increase the likelihood of developing an app that is testable, useable, and visually appealing on an iOS device. 
	Considering all of the options we have for building a native iOS application, it makes sense to choose Xcode and Swift. 
	
	\section{iOS User Interface Organization}
	\subsection{Option \#1 - replace with name of technology}
	\subsection{Option \#2}
	\subsection{Option \#3}
	\subsection{Goals}
	\subsection{Criteria Evaluation}
	e.g. cost, availability, speed, security.
	\subsection{Table Comparison}
	\begin{table}[ht]
	\caption{Comparison of iOS UI Organization}
	\begin{center}
	\begin{tabular} { | m{3cm} | m{5cm} | m{5cm} | }
	\hline\hline
	Tool & Brief description & Rationale \\ [0.5ex]
	%heading
	\hline
	Row 1 Col 1 & Col 2 & Col 3 \\
	\hline
	Row 2 Col 1 & Col 2 & Col 3 \\
	\hline
	Row 3 Col 1 & Col 2 & Col 3 \\
	\hline
	\end{tabular}
	\end{center}
	\end{table}
	\subsection{Discussion}
	\subsection{Selection}
	
	\section{Android Development Platform}
	\subsection{Option \#1 - replace with name of technology}
	\subsection{Option \#2}
	\subsection{Option \#3}
	\subsection{Goals}
	\subsection{Criteria Evaluation}
	e.g. cost, availability, speed, security.
	\subsection{Table Comparison}
	\begin{table}[ht]
	\caption{Comparison of Android Development Platforms}
	\begin{center}
	\begin{tabular} { | m{3cm} | m{5cm} | m{5cm} | }
	\hline\hline
	Tool & Brief description & Rationale \\ [0.5ex]
	%heading
	\hline
	Row 1 Col 1 & Col 2 & Col 3 \\
	\hline
	Row 2 Col 1 & Col 2 & Col 3 \\
	\hline
	Row 3 Col 1 & Col 2 & Col 3 \\
	\hline
	\end{tabular}
	\end{center}
	\end{table}
	\subsection{Discussion}
	\subsection{Selection}
	
	\section{Android User Interface Organization}
	\subsection{Option \#1 - replace with name of technology}
	\subsection{Option \#2}
	\subsection{Option \#3}
	\subsection{Goals}
	\subsection{Criteria Evaluation}
	e.g. cost, availability, speed, security.
	\subsection{Table Comparison}
	\begin{table}[ht]
	\caption{Comparison of Android UI Organization}
	\begin{center}
	\begin{tabular} { | m{3cm} | m{5cm} | m{5cm} | }
	\hline\hline
	Tool & Brief description & Rationale \\ [0.5ex]
	%heading
	\hline
	Row 1 Col 1 & Col 2 & Col 3 \\
	\hline
	Row 2 Col 1 & Col 2 & Col 3 \\
	\hline
	Row 3 Col 1 & Col 2 & Col 3 \\
	\hline
	\end{tabular}
	\end{center}
	\end{table}
	\subsection{Discussion}
	\subsection{Selection}
	
	\section{Cross-Platform Development}
	\subsection{Option \#1 - replace with name of technology}
	\subsection{Option \#2}
	\subsection{Option \#3}
	\subsection{Goals}
	\subsection{Criteria Evaluation}
	e.g. cost, availability, speed, security.
	\subsection{Table Comparison}
	\begin{table}[ht]
	\caption{Comparison of Cross-Platform Development}
	\begin{center}
	\begin{tabular} { | m{3cm} | m{5cm} | m{5cm} | }
	\hline\hline
	Tool & Brief description & Rationale \\ [0.5ex]
	%heading
	\hline
	Row 1 Col 1 & Col 2 & Col 3 \\
	\hline
	Row 2 Col 1 & Col 2 & Col 3 \\
	\hline
	Row 3 Col 1 & Col 2 & Col 3 \\
	\hline
	\end{tabular}
	\end{center}
	\end{table}
	\subsection{Discussion}
	\subsection{Selection}
	
	\section{Integrating the Giving Platform}
	\subsection{Option \#1 - replace with name of technology}
	\subsection{Option \#2}
	\subsection{Option \#3}
	\subsection{Goals}
	\subsection{Criteria Evaluation}
	e.g. cost, availability, speed, security.
	\subsection{Table Comparison}
	\begin{table}[ht]
	\caption{Comparison of Giving Platforms}
	\begin{center}
	\begin{tabular} { | m{3cm} | m{5cm} | m{5cm} | }
	\hline\hline
	Tool & Brief description & Rationale \\ [0.5ex]
	%heading
	\hline
	Row 1 Col 1 & Col 2 & Col 3 \\
	\hline
	Row 2 Col 1 & Col 2 & Col 3 \\
	\hline
	Row 3 Col 1 & Col 2 & Col 3 \\
	\hline
	\end{tabular}
	\end{center}
	\end{table}
	\subsection{Discussion}
	\subsection{Selection}
	
	\section{Integrating E-Bulletins}
	\subsection{Option \#1 - replace with name of technology}
	\subsection{Option \#2}
	\subsection{Option \#3}
	\subsection{Goals}
	\subsection{Criteria Evaluation}
	e.g. cost, availability, speed, security.
	\subsection{Table Comparison}
	\begin{table}[ht]
	\caption{Comparison of E-Bulletin Integration}
	\begin{center}
	\begin{tabular} { | m{3cm} | m{5cm} | m{5cm} | }
	\hline\hline
	Tool & Brief description & Rationale \\ [0.5ex]
	%heading
	\hline
	Row 1 Col 1 & Col 2 & Col 3 \\
	\hline
	Row 2 Col 1 & Col 2 & Col 3 \\
	\hline
	Row 3 Col 1 & Col 2 & Col 3 \\
	\hline
	\end{tabular}
	\end{center}
	\end{table}
	\subsection{Discussion}
	\subsection{Selection}
	
	\section{Integrating the Sermon Platform}
	\subsection{Option \#1 - replace with name of technology}
	\subsection{Option \#2}
	\subsection{Option \#3}
	\subsection{Goals}
	\subsection{Criteria Evaluation}
	e.g. cost, availability, speed, security.
	\subsection{Table Comparison}
	\begin{table}[ht]
	\caption{Comparison of Sermon Platforms}
	\begin{center}
	\begin{tabular} { | m{3cm} | m{5cm} | m{5cm} | }
	\hline\hline
	Tool & Brief description & Rationale \\ [0.5ex]
	%heading
	\hline
	Row 1 Col 1 & Col 2 & Col 3 \\
	\hline
	Row 2 Col 1 & Col 2 & Col 3 \\
	\hline
	Row 3 Col 1 & Col 2 & Col 3 \\
	\hline
	\end{tabular}
	\end{center}
	\end{table}
	\subsection{Discussion}
	\subsection{Selection}
	
	\section{Integrating the Calendar Platform}
	\subsection{Option \#1 - replace with name of technology}
	\subsection{Option \#2}
	\subsection{Option \#3}
	\subsection{Goals}
	\subsection{Criteria Evaluation}
	e.g. cost, availability, speed, security.
	\subsection{Table Comparison}
	\begin{table}[ht]
	\caption{Comparison of Calendar Platforms}
	\begin{center}
	\begin{tabular} { | m{3cm} | m{5cm} | m{5cm} | }
	\hline\hline
	Tool & Brief description & Rationale \\ [0.5ex]
	%heading
	\hline
	Row 1 Col 1 & Col 2 & Col 3 \\
	\hline
	Row 2 Col 1 & Col 2 & Col 3 \\
	\hline
	Row 3 Col 1 & Col 2 & Col 3 \\
	\hline
	\end{tabular}
	\end{center}
	\end{table}
	\subsection{Discussion}
	\subsection{Selection}
	
	\section{Conclusion}
	
	\bibliographystyle{ieeetr}
	\bibliography{techbib}

	


\end{document}
