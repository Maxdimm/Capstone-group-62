\documentclass[letterpaper,10pt,draftclsnofoot,onecolumn,titlepage]{IEEEtran}

\usepackage{graphicx}
\usepackage{amssymb}
\usepackage{amsmath}
\usepackage{amsthm}
\usepackage{alltt}
\usepackage{float}
\usepackage{color}
\usepackage{url}
\usepackage{enumitem}
\usepackage{pstricks, pst-node}
\usepackage{geometry}
\usepackage{array}


\geometry{margin = .75in}

\usepackage{hyperref}

\newcommand*{\signature}[1]{%
	\par\noindent\makebox[3.5in]{\hrulefill} \hfill\makebox[3.0in]{\hrulefill}%
	\par\noindent\makebox[3.5in][l]{#1}	    \hfill\makebox[3.0in][l]{Date}%
}%

\def\name{Kevin Stine, Courtney Bonn, Maxwell Dimm}
\def\team{Calvary Chapel Corvallis}

\hypersetup{
	colorlinks = true,
	urlcolor = black,
	linkcolor = black,
	pdfauthor = {\name},
	pdftitle = {CS461 Tech Review},
	pdfsubject = {CS461 Tech Review},
	pdfpagemode = UseNone
}

\begin{document}
	\title{\huge \team \\ Technology Review \\ CS 461 Fall 2016}
	\author{\large \name}

	\maketitle
		\begin{abstract}The purpose of this project is to produce an iOS/Android application for Calvary Chapel of Corvallis that will allow members to access a plethora of information all in one localized space.
		The Church's current website doesn't provide an interface where current members of the church can very quickly access important information such as events, bulletins, and messages from the service.
		The desired application will be simple enough for anyone to use while providing back end access for staff to easily upload new information to the app.
		The priorities lie in maximizing the usability of the app and providing bulletin, schedule, video, and giving functionality.
		We will work with the existing Calvary Chapel web development team to create a product that is seamlessly integrated with their already existing network.
		\end{abstract}


	\vspace{2in}
	\signature{Client}
	\vspace{.25in}
	\signature{Courtney Bonn}
	\vspace{.25in}
	\signature{Kevin Stine}
	\vspace{.25in}
	\signature{Maxwell Dimm}

	\clearpage

	\tableofcontents

	\clearpage

	\section{Introduction}
	This document will break down our application in 9 different parts.
	There will be an outline on 3 different types of technology that can be used to implement the 9 different parts of the system.
	We will detail each type of technology as well as discuss why it could be useful for the project.
	We will then compare each type of technology against each other and analyze which one would be the best for us to use in production.
	The 9 pieces of the app are divided among the three group members.
	iOS development platform, iOS user interface organization, and integrating the e-bulletins from the current website will be researched and handled by Courtney Bonn.
	Android development platform, Android user interface organization, and integrating the current giving platform will be researched and handled by Kevin Stine.
	Finally, cross-platform development,  integrating the recent sermons, and integrating the existing calendar will be researched and handled by Max Dimm.


	\section{iOS Development Platform}
	\subsection{Xcode using Swift}
	The main technology used to develop iOS applications is Xcode.
	Xcode is a program that can be downloaded on any Mac OS platform and isn't able to be downloaded on Windows PCs.
	It makes use of developer tools to allow users to create an iOS app that will work with all updated iOS devices.
	This tool uses the language of Swift, which is a new programming language that has been released by Apple \cite{AppleSwift}.
	For people who have never developed an app before, Swift is considered the easier language to learn \cite{CodeChris}.
	The tool allows the developer to use a simulator that will display the graphical interface that the user will see and makes the process of developing an app for iOS devices go smoothly.

	\subsection{AppCode}
	AppCode is a smart IDE that is used for iOS/Mac OS development \cite{AppCode}.
	There are many features for this IDE that allows app developers to easily and quickly develop an application without dealing with any hassle from their IDE.
	Some of the features include efficient project navigation, smart completion, reliable refactorings, thorough code analysis, and unit testing.
	Additionally, AppCode not only supports Swift, but Objective-C, C, C++, JavasScript, XML, HTML, CSS, and XPath.
	This wide variety of languages allows this IDE to be used for more than just iOS development all in one program.
	AppCode also works directly with Xcode so developing an app for iOS devices is made simple.
	This IDE is a paid subscription, costing \$199 for the first year and the price decreasing after that.
	There is a free 30-day trial available to determine whether or not the program is the right tool for the wanted product.
	\subsection{iPhone App Builder}
	The third option for a toolkit to build an iOS app is a program that will essentially build the app for you, leaving no need to learn how to code.
	One program, AppyPie \cite{AppyPie}, boasts that it will take all of the work out of creating an app and allow an unskilled user to get a working app quickly uploading to the App store.
	This option would be useful for people who have no prior programming experience and are just looking to get an app produced with little effort.
	\subsection{Goals}
	The goal for comparing iOS development tool is to acquire which way is most acceptable and makes most sense for developing an iOS application.
	We want to determine which option is relevant to current applications and which option will produce a valid application.
	\subsection{Criteria Evaluation}
	The criteria used for each option is relevancy to current application development, ease of which one can learn how to use the programming language, and the level of difficulty one is looking for when developing the app.
	\subsection{Table Comparison}

	\begin{table}[ht]
	\caption{Comparison of iOS Development Platforms}
	\begin{center}
	\begin{tabular} { | m{3cm} | m{5cm} | m{5cm} | }
	\hline\hline
	Tool & Brief description & Rationale \\ [0.5ex]
	%heading
	\hline
	Xcode using Swift & Using the Swift programming language to write the application using the program Xcode & Best option as it's the most recent and will allow us to actually learn how to develop an app \\
	\hline
	AppCode & Using the AppCode IDE to write an iOS app & Great option if price weren't an issue as it does incorporate multiple languages into the IDE \\
	\hline
	iPhone App Builder & Using a third-party program to automatically build the app & A good option for people who have no programming experience and just want to quickly produce an app \\
	\hline
	\end{tabular}
	\end{center}
	\end{table}

	\subsection{Discussion}
	Using Xcode with Swift is a strong option for building iOS--some would say that it is one of the only options.
	Using a third-party IDE like AppCode is also a strong option as it does have many other features, like the ability to program in other language, that Xcode doesn't have. Not only that, it does integrate itself with Xcode. However, at \$199 for 1 year, the price is steep and may only make sense for those who develop apps professionally.

	The idea of using a third-party program to build the app automatically is tempting, but would defeat the purpose of learning how to develop an iOS app.
	If the objective of this project was to produce an iOS app within a matter of days or weeks, then this option could be considered useful.

	\subsection{Selection}
	Because our main objective for this project is to learn the detail of producing an iOS app, it is best to choose Xcode with Swift.
	Swift is the most updated language for iOS development and using Xcode simplifies the process of building the application.
	With Xcode and Swift, we will be able to properly produce an application that can be submitted to the App Store for download.
	It will increase the likelihood of developing an app that is testable, usable, and visually appealing on an iOS device.
	Considering all of the options we have for building a native iOS application, it makes sense to choose Xcode and Swift.

	\section{iOS User Interface Organization}
	\subsection{Sketch}
	Sketch is a paid program exclusively available on Mac OS X 10.10+ \cite{Sketch}.
	This program gives app designers a platform for organizing the user interface.
	There are many features that are available in this program, including precision, objects, the inspector, tools, reusable elements, and exporting.
	Precision allows scalability to be present in the app.
	Objects allow each shape created to be easily findable and completely editable.
	The inspector is the tool for dimensions, positioning, opacity and everything else you'd need in order to control the design of the app.
	Because Sketch is exclusively available on Mac, it uses Apple's frameworks so the app will be completely supported.
	Most importantly, there is a Sketch app available for download on iOS devices which allows the designer to view the app design on an actual iOS device, rather than just on the computer.
	This program is \$99.00 but there is a free trial and education discounts.
	\subsection{Interface Builder}
	Within the Mac native program Xcode, there is a built-in editor called Interface Builder that allows the designer to build a user interface \cite{Interface}.
	It uses a Model-View-Controller pattern which allows the designer to build the interface without worrying about the implementation of the actual app.
	This program will connect the user interface to the code for the app automatically.
	There are multiple views in an app, and this builder uses storyboards to keep the multiple views organized.
	This tool gives the user the option to preview the user interface without actually running the app which saves time.
	Users are automatically given access to the Interface Builder when downloading and using Xcode.
	\subsection{Option \#3}
	\subsection{Goals}
	\subsection{Criteria Evaluation}
	e.g. cost, availability, speed, security.
	\subsection{Table Comparison}
	\begin{table}[ht]
	\caption{Comparison of iOS UI Organization}
	\begin{center}
	\begin{tabular} { | m{3cm} | m{5cm} | m{5cm} | }
	\hline\hline
	Tool & Brief description & Rationale \\ [0.5ex]
	%heading
	\hline
	Row 1 Col 1 & Col 2 & Col 3 \\
	\hline
	Row 2 Col 1 & Col 2 & Col 3 \\
	\hline
	Row 3 Col 1 & Col 2 & Col 3 \\
	\hline
	\end{tabular}
	\end{center}
	\end{table}
	\subsection{Discussion}
	\subsection{Selection}

	\section{Android Development Platform}
		\subsection{Android Studio}
			Android Studio is the official integrated development environment (IDE) for Android.
			It provides the fastest tools for building apps on every type of Android device.
			Android Studio comes with tons of tools that will make development for Android much easier.
			The IDE has a feature called Instant Run, which allows you to push code and resource changes to your app while it's running on a device or emulator to see the changes instantly come to life.
			This feature can drastically help speed up development, as we won't have to constantly rebuild the app and start up the emulator each time we want to test a new change.
			Android Studio also features an intelligent code editor, allowing us to write better code, work faster and be more productive.
			Android Studio is built on IntelliJ and is capable of advanced code completion, refactoring and code analysis.
			It also has a fast emulator which will allow us to quickly get our app up and running for testing and debugging.
			Android Studio also has the following features: a flexible Gradle-based build system, GitHub integration, extensive testing tools and frameworks and lint tools to catch performance, usability, version compatibility and other problems.

		\subsection{Appcelerator}
			Appcelerator is a platform which provides everything you need to create native mobile applications - all from a single JavaScript code base.
			Appcelerator offers direct access to native APIs using Hyperloop, delivers fully native apps for rich user experience, immediate support for each new OS release, and seamless integration to existing continuous delivery systems.
			Appcelerator also comes bundled with your own MBaaS (Mobile Backend as a service).
			This feature, called Arrow, is a powerful opinionated framework for building and running APIs.
			Arrow would allow us to deliver data to any app client, engage users with pre-built notification capability, trigger notifications based on user location or predetermined schedule and view notification history and details.
			This could be extremely helpful for the utilization of push notifications for Church members.
			We could have schedules that would send out a notification at 10:15 AM with the morning bulletin for users that are at church that morning.
			This would be very useful as we wouldn't be sending out notifications to every app user since not everyone might be able to attend that particular day.
			Appcelerator also comes with real-time mobile analytics for every native app - whether built on the Appcelerator Platform or directly via native SDK (iOS \& Android).
			This provides a mobile lifecycle dashboard for visibility into all apps and performance and crash analytics.
			This feature could be extremely helpful for the church so they can monitor who is using the app and get some visual feedback on it's success.

		\subsection{Adobe PhoneGap}
			PhoneGap is a platform used to create mobile apps that are powered by open web technology.
			PhoneGap allows you quickly make hybrid applications build with HTML, CSS and JavaScript.
			It also allows you to create experiences for multiple platforms from a single codebase so you can reach your audience no matter their device.
			PhoneGap includes PhoneGap Build, which takes the pain out of compiling PhoneGap apps.
			Build allows you to get app-store ready apps without maintaining native SDK's and compiles it for you in the cloud.
			PhoneGap includes a desktop app for creating apps without using the command line, and a mobile app to connect your device to your development machine to see changes instantly.
			This could be a great resource to utilize since it's free, and the apps are created using HTML, CSS and JavaScript rather than Java.

		\subsection{Goals}
			The goal is to determine which development platform we would like to utilize for creating our Android version of the app.
			Since we need to create an app for both Android and iOS, it's important to determine how we want to go about that.
			We can create applications using the native SDK's through Android Studio and Java, or we could utilize a third party application such as Appcelerator or PhoneGap which will work using HTML, CSS and JavaScript.
			The goal is to determine which method we think would be most effective for creating our application.

		\subsection{Criteria Evaluation}
			The criteria for determining which development platform we use will be based off:
			\begin{enumerate}
				\item \textbf{Usability:} Is the development platform user friendly, have the correct tools we need to utilize and allow for quick development.
				\item \textbf{Price:} Is the development platform free and open source, is there an upfront one time cost, or a recurring cost.
				\item \textbf{Language:} Does the development platform utilize programming languages which we are already familiar with, or will we have to learn a new language to create a mobile app.
			\end{enumerate}

		\subsection{Table Comparison}
		\begin{table}[ht]
			\caption{Comparison of Android Development Platforms}
			\begin{center}
				\begin{tabular} { | m{3cm} | m{5cm} | m{5cm} | }
					\hline\hline
					Tool & Brief description & Rationale \\ [0.5ex]
					%heading
					\hline
					Android Studio & Using Android Studio and the native Android SDK & Best option for creating a native app built in Java that will be platform specific \\
					\hline
					Appcelerator & Using the Appcelerator Platform to create an iOS and Android app in JavaScript & Best option for the number of features, can be written in JavaScript but is not free\\
					\hline
					Adobe PhoneGap & Using the Adobe PhoneGap platform to creat an iOS and Android app build in HTML, CSS and JavaScript & Great option for usability since we can utilize previous Web Development skill \\
					\hline
				\end{tabular}
			\end{center}
		\end{table}

		\subsection{Discussion}
			Android Studio offers the most pure Android development experience as it utilizes the native Android SDK.
			This makes it a strong contender for the platform we will choose for development.
			In addition to utilizing the Android SDK, it has features such as Instant Run which would allow us to constantly make changes to the app while it is running to see the changes occur instantly.
			This makes Android Studio very powerful and helpful through the development process.

			Appcelerator is another strong contender as it provides a ton of functionality since it comes bundled with a few other features such as Titanium, Arrow and user analytics.
			While this could provide a lot of good information for the church once the app is up and running, it does cost roughly \$400 dollars a year which is a bit pricey.
			To get the most out of this platform and the best return on investment, it would require the church to have someone using all these features by actively monitoring analytics and crash reports.
			We think that even though this looks like it's an incredibly poewrful platform, it would be underutilized and provide more functionality than needed for the scope of this app.

			Adobe PhoneGap does seem like a very promising option as it allows the app to be developed in HTML, CSS and JavaScript.
			This is a plus since our team is already familiar with these languages and there would be less of a learning curve then utilizing Java.
			PhoneGap comes with a desktop app and mobile app for easy development.
			Rather than having to deal with compiling the application PhoneGap handles that in the cloud.
			This could be a really good option as it might be simpler to use with less of a learning curve than other platforms.

		\subsection{Selection}
			Based on the scope of this project, we've decided that utilizing Android Studio with the native Android SDK will be the best option moving forward.
			Since the team joined this project in hopes of learning more about mobile app development, we think that utilizing the native Android SDK will give us the best learning experience.
			While developing the app in Java may have a steeper learning curve than HTML or JavaScript, we believe that we will learn the most by using the language that most Android apps are written in.
			We hope that we will also be able to create a smooth and well polished app using Android Studio and features that it comes with.
			As Android Studio if the official IDE for Android, we believe it is the best platform to use.

	\section{Android User Interface Organization}
		\subsection{Tab Navigation}
			There are many different ways in which we can structure our Android app.
			Navigating between pages and windows is a large part of the User Interface, and we want to make sure we are using the most intuitive method for navigation.
			One method would be to use tabs which make it easy to explore and switch between different views.
			Tabs enable content organization at a high level and can be used for switching between views, data sets, or functional aspects of an app. \cite{Material-Tabs}
			The utilization of tabs would allow for swipe gestures between the various tabs of the application.
			Because it uses swipe gestures, we would need to make sure to not use content which also supports swiping.
			Tabs provide an easy to use and recognizable view which immediately lets users know that they can swipe between the content listed at the top.
			Tabs are a good choice because it provides all of the content directly on the first page for users to see.
			This would prevent users from getting confused since the navigation pane won't be hidden.

		\subsection{Side Bar Navigation}
			Side bar navigation is another UI layout which creates a navigation bar to the left of the content.
			Side nav bars can be pinned for permanent display or can float temporarily as overlays. \cite{Material-Side-Nav}
			Temporary nav drawers overlay the content canvas, which is likely how we would use a side nav bar in our application.
			Side nav bars are an excellent option for navigation since it essentially hides the navigation pane until the user swipes in from the left of their screen.
			This allows for the focus of the page to be on the content and would allow us to maximize the space that the user interacts with.
			Instead of taking up screen space for a navigation pane, we can hide the navigation panel and have it accessible to the user from any page.
			Side nav bars are extremely popular in most material design inspired applications, and would fit perfectly in our app as well.

		\subsection{Bottom Navigation}
			Bottom navigation make it easy to explore and switch between top-level views in a single tap.
			Tapping on a bottom navigation icon takes you directly to the associated view or refreshes the currently active view. \cite{Material-Bottom-Nav}
			This is a pretty standard UI layout that can be found on the majority of iOS applications.
			This would be a great choice for being able to match the layout we use for our iOS application.
			All of the pages are neatly laid out at the bottom of the screen, so it's easy for the user to know exactly which pages are available and how to access them.
			The only downside of this layout would be for Android phones that have on-screen soft buttons rather than off-screen hardware buttons.
			This could potentially provide an issue for users trying to access certain content but rather then clicking the page they want, they accidentily click the home button instead.
			Despite this, bottom navigation would be a great option for continuity across both of our mobile applications.

		\subsection{Goals}
			Our goal is to determine which UI layout we want to use for our Android application.
			We want to narrow down the three main UI layouts to determine which one will be most user friendly and most intuitive to app users.
		\subsection{Criteria Evaluation}
			The criteria for determining which UI layout we use will be based off:
			\begin{enumerate}
				\item \textbf{Continuity:} Does this layout provide a similar look and feel to our iOS application.
				\item \textbf{Usability:} Is this layout easy to use and does it provide enough context for the user to know how to navigate the application.
				\item \textbf{Intuitiveness:} Is this layout simple and easier for anyone to understand and are they able to navigate the various pages without being walked through the process.
			\end{enumerate}

		\subsection{Table Comparison}
			\begin{table}[ht]
			\caption{Comparison of Android UI Organization}
				\begin{center}
				\begin{tabular} { | m{3cm} | m{5cm} | m{5cm} | }
					\hline\hline
					Tool & Brief description & Rationale \\ [0.5ex]
					%heading
					\hline
					Tab Navigation & Layout that includes tabs at the top of the page & A great option for being able to quickly swipe between pages \\
					\hline
					Side Bar Navigation & Layout that includes a nav drawer overlay & A great option to have a hidden navigation bar that is accessible from any page \\
					\hline
					Bottom Navigation & Layout that includes bottom buttons & Col 3 A great option for continuity with our iOS app and easily viewable information\\
					\hline
				\end{tabular}
				\end{center}
			\end{table}

		\subsection{Discussion}
			The use of tabs for the navigation of the app is a very intuitive and easy layout for people that are familiar with Android devices.
			On Android, almost everything can be done by a simple swipe motion.
			This is less common on iOS devices, so it could be confusing for people that are used to iOS instead of Android.
			This is also true of the sidebar navigation pane, which is hidden be default unless the user swipes in from the left or clicks on the hamburger menu.
			For continuity, the use of the bottom navigation would make the most sense as a user switching from iOS to Android will find it similar.
			This does however cause issues with some Android users that have on-screen soft buttons.

		\subsection{Selection}
			While the sidebar can be a little less intuitive than other UI layouts, we think this will be the best option to use in our application.
			With the navbar off to the left side of the page, it will allow us to focus on the clarity of the content that we are presenting to the user.
			This frees up screen space by not using it for tabs or buttons.
			This also allows the user to easily get to the settings and access all other pages by swiping in from the left.
			We believe that the side navigation bar will be the most intuitive and easiest to use once people use it for the first time.


	\section{Cross-Platform Development}
	\subsection{Option \#1 - replace with name of technology}
	\subsection{Option \#2}
	\subsection{Option \#3}
	\subsection{Goals}
	\subsection{Criteria Evaluation}
	e.g. cost, availability, speed, security.
	\subsection{Table Comparison}
	\begin{table}[ht]
	\caption{Comparison of Cross-Platform Development}
	\begin{center}
	\begin{tabular} { | m{3cm} | m{5cm} | m{5cm} | }
	\hline\hline
	Tool & Brief description & Rationale \\ [0.5ex]
	%heading
	\hline
	Row 1 Col 1 & Col 2 & Col 3 \\
	\hline
	Row 2 Col 1 & Col 2 & Col 3 \\
	\hline
	Row 3 Col 1 & Col 2 & Col 3 \\
	\hline
	\end{tabular}
	\end{center}
	\end{table}
	\subsection{Discussion}
	\subsection{Selection}

	\section{Integrating the Giving Platform}
	\subsection{Option \#1 - replace with name of technology}
	\subsection{Option \#2}
	\subsection{Option \#3}
	\subsection{Goals}
	\subsection{Criteria Evaluation}
	e.g. cost, availability, speed, security.
	\subsection{Table Comparison}
	\begin{table}[ht]
	\caption{Comparison of Giving Platforms}
	\begin{center}
	\begin{tabular} { | m{3cm} | m{5cm} | m{5cm} | }
	\hline\hline
	Tool & Brief description & Rationale \\ [0.5ex]
	%heading
	\hline
	Row 1 Col 1 & Col 2 & Col 3 \\
	\hline
	Row 2 Col 1 & Col 2 & Col 3 \\
	\hline
	Row 3 Col 1 & Col 2 & Col 3 \\
	\hline
	\end{tabular}
	\end{center}
	\end{table}
	\subsection{Discussion}
	\subsection{Selection}

	\section{Integrating E-Bulletins}
	\subsection{Option \#1 - replace with name of technology}
	\subsection{Option \#2}
	\subsection{Option \#3}
	\subsection{Goals}
	\subsection{Criteria Evaluation}
	e.g. cost, availability, speed, security.
	\subsection{Table Comparison}
	\begin{table}[ht]
	\caption{Comparison of E-Bulletin Integration}
	\begin{center}
	\begin{tabular} { | m{3cm} | m{5cm} | m{5cm} | }
	\hline\hline
	Tool & Brief description & Rationale \\ [0.5ex]
	%heading
	\hline
	Row 1 Col 1 & Col 2 & Col 3 \\
	\hline
	Row 2 Col 1 & Col 2 & Col 3 \\
	\hline
	Row 3 Col 1 & Col 2 & Col 3 \\
	\hline
	\end{tabular}
	\end{center}
	\end{table}
	\subsection{Discussion}
	\subsection{Selection}

	\section{Integrating the Sermon Platform}
	\subsection{Option \#1 - replace with name of technology}
	\subsection{Option \#2}
	\subsection{Option \#3}
	\subsection{Goals}
	\subsection{Criteria Evaluation}
	e.g. cost, availability, speed, security.
	\subsection{Table Comparison}
	\begin{table}[ht]
	\caption{Comparison of Sermon Platforms}
	\begin{center}
	\begin{tabular} { | m{3cm} | m{5cm} | m{5cm} | }
	\hline\hline
	Tool & Brief description & Rationale \\ [0.5ex]
	%heading
	\hline
	Row 1 Col 1 & Col 2 & Col 3 \\
	\hline
	Row 2 Col 1 & Col 2 & Col 3 \\
	\hline
	Row 3 Col 1 & Col 2 & Col 3 \\
	\hline
	\end{tabular}
	\end{center}
	\end{table}
	\subsection{Discussion}
	\subsection{Selection}

	\section{Integrating the Calendar Platform}
	\subsection{Option \#1 - replace with name of technology}
	\subsection{Option \#2}
	\subsection{Option \#3}
	\subsection{Goals}
	\subsection{Criteria Evaluation}
	e.g. cost, availability, speed, security.
	\subsection{Table Comparison}
	\begin{table}[ht]
	\caption{Comparison of Calendar Platforms}
	\begin{center}
	\begin{tabular} { | m{3cm} | m{5cm} | m{5cm} | }
	\hline\hline
	Tool & Brief description & Rationale \\ [0.5ex]
	%heading
	\hline
	Row 1 Col 1 & Col 2 & Col 3 \\
	\hline
	Row 2 Col 1 & Col 2 & Col 3 \\
	\hline
	Row 3 Col 1 & Col 2 & Col 3 \\
	\hline
	\end{tabular}
	\end{center}
	\end{table}
	\subsection{Discussion}
	\subsection{Selection}
	\clearpage
	\section{Conclusion}

	\bibliographystyle{ieeetr}
	\bibliography{techbib}




\end{document}
