\documentclass[letterpaper,10pt,draftclsnofoot,onecolumn,titlepage]{IEEEtran}

\usepackage{graphicx}
\usepackage{amssymb}
\usepackage{amsmath}
\usepackage{amsthm}
\usepackage{alltt}
\usepackage{float}
\usepackage{color}
\usepackage{url}
\usepackage{enumitem}
\usepackage{pstricks, pst-node}
\usepackage{geometry}

\geometry{margin = .75in}

\usepackage{hyperref}

\def\name{Kevin Stine, Courtney Bonn, Maxwell Dimm}

\hypersetup{
	colorlinks = true,
	urlcolor = black,
	pdfauthor = {\name},
	pdftitle = {CS461 Requirements},
	pdfsubject = {CS461 Requirements},
	pdfpagemode = UseNone
}

\begin{document}
	\title{\huge Requirements \\ CS 461 Fall 2016}
	\author{\large \name}
	
	\maketitle
		\begin{abstract}The purpose of this project is to produce an iOS/Android application for Calvary Chapel of Corvallis that will allow members to access a plethora of information all in one localized space. 
		The Church's current website doesn't provide an interface where current members of the church can very quickly access important information such as events, bulletins, and messages from the service. 
		The desired application will be simple enough for anyone to use while providing back end access for staff to easily upload new information to the app. 
		The priorities lie in maximizing the usability of the app and providing bulletin, schedule, video, and donation functionality. 
		We will work with the existing Calvary Chapel web development team to create a product that is seamlessly integrated with their already existing network. 
		\end{abstract}
	
	\clearpage	
		
	\section{Introduction}
	\subsection{Purpose}
	The purpose of this document is to list, in detail, all of the requirements intended for our project. 
	Additionally, there will be an overall description of the project, explaining the different aspects of the application in which we are building. 
	This document is intended for the client team at Calvary Corvallis Church, as well as the teachers and teachers assistants for our class. 
	
	\subsection{Scope}
	At the completion of this project, the software product to be produced is called The Calvary Corvallis Church App (placeholder for official name). 
	This application will provide a portion of the information already available on the church website in a mobile-friendly application.
	The application will not be an exact replica of the website, but rather a combination of certain sections that are used heavily by current members of the church. 
	The main goal of the application is to offer a space for the active members of the church to stay current with events, messages, and the e-bulletin. 
	It will benefit the church members in a different way than their current website, because the application will allow them to have all of the necessary information saved on their smartphone, rather than opening a new browser each time they need to access the website. 
	In order to ensure this application is available to all members, the application will be offered for both iOS and Android users. 
	
	\subsection{Definitions, acronyms, and abbreviations}
	
	\subsection{References}
	
	\subsection{Overview}
	Describe what the rest of the document contains. 
	Explains how the document is organized. 
	
	\section{Overall description}
	\subsection{Product perspective}
	Put the product into perspective with other related products. 
	If the product is a component of a larger system, then relate the requirements of that larger system to this software.
	
	\subsubsection{System interfaces}
	List each system interface and identity the functionality of the software to accomplish the system requirement and the interface description to match the system.
	
	\subsubsection{User interfaces}
	Specify the logical characteristics of each interface between the product and its users (includes required screen formats, page or window layouts, content of reports or menus, availability of programmable function keys)
	Specify all the aspects of optimizing the interface with the person who must use the system. 
	Consider error messages and how those might appear to the user.
	
	\subsubsection{Hardware interfaces}
	Devices that are supported, how they are supported, protocols. 
	Logical characteristics between the software product and hardware components.
	
	\subsubsection{Software interfaces}
	Specify the use of other required software products and interfaces with other application systems.
	Include: name, menumonic, specification number, version number source, for each required software product. 
	Include: discussion of the purpose of the interfacing software as related to the product, definition of the interface in terms of message content and format, for each interface.
	
	\subsubsection{Communications interfaces}
	Specify the various interfaces to communications such as local network protocols.
	
	\subsubsection{Memory}
	Specify any applicable characteristics and limits on primary and secondary memory.
	
	\subsubsection{Operations}
	Specify normal and special operations required by the user; various modes of operations; periods of interactive operations and periods of unattended operations; data processing support functions; backup and recovery operations.
	Can be specified in the user interfaces section. 
	
	\subsubsection{Site adaptation requirements}
	Define requirements for data or sequences specific to a given site, mission, or operational mode.
		
	\subsection{Product functions}
	Provide a summary of the major functions the software will perform.
	
	\subsection{User characteristics}
	Describe the general characteristics of the intended users of the product including educational level, experience, and technical expertise. 
	
	\subsection{Constraints}
	Provide a general description of anything else that will limit the developer's options such as: regulatory policies, hardware limitations, interfaces to other applications, parallel operation, audit functions, control functions, higher-order language requirements, signal handshake protocols, reliability requirements, criticality of the application, safety and security considerations.
	
	\subsection{Assumptions and dependencies} 
	List each factor that affect the requirements. 
	Any changes that can affect the requirements. 
	For example, an assumption may be that a specific OS will be available on the hardware designated for the product, but if it's not available the SS would have to change.
	
	\subsection{Apportioning of requirements}
	Identify requirements that may be delayed until future versions of the system.
	
	\section{Specific requirements}
	\subsection{External Interfaces}
	Detailed description of all inputs into and outputs from the software system. Don't repeat info from previous section.
	
	\subsection{Functions}
	Define fundamental actions that must take place in the software.
	
	\subsection{Performance requirements}
	Specify both the static and dynamic numerical requirements placed on the software or on human interaction with the software a whole.
	Specify number of terminals to be supported, number of simultaneous users, amount and type of information to be handled. 
	
	\subsection{Logical database requirements}
	Specify logical requirements for any information to be placed into a database.
	
	\subsection{Design constraints}
	Specify constraints imposed by other standards, hardware limitations, etc.
	
	\subsubsection{Standards compliance}
	Specify requirements derived from existing standards or regulations. 
	
	\subsection{Software system attributes}
	\subsubsection{Reliability}
	\subsubsection{Availability}
	\subsubsection{Security}
	\subsubsection{Maintainability}
	\subsubsection{Portability}
	
	\subsection{Organizing the specific requirements}
	**not sure if we need this section
	
	\section{Supporting Information}
	\subsection{Table of contents and index}
	\subsection{Appendixes}
 
\end{document}
