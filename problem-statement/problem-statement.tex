\documentclass[letterpaper,10pt,draftclsnofoot,onecolumn,titlepage]{IEEEtran}

\usepackage{graphicx}
\usepackage{amssymb}
\usepackage{amsmath}
\usepackage{amsthm}
\usepackage{alltt}
\usepackage{float}
\usepackage{color}
\usepackage{url}
\usepackage{enumitem}
\usepackage{pstricks, pst-node}
\usepackage{geometry}

\geometry{margin = .75in}

\usepackage{hyperref}

\newcommand*{\signature}[1]{%
	\par\noindent\makebox[3.5in]{\hrulefill} \hfill\makebox[3.0in]{\hrulefill}%
	\par\noindent\makebox[3.5in][l]{#1}	    \hfill\makebox[3.0in][l]{Date}%
}%

\def\name{Kevin Stine, Courtney Bonn, Maxwell Dimm}

\hypersetup{
	colorlinks = true,
	urlcolor = black,
	pdfauthor = {\name},
	pdftitle = {CS461 Problem Statement},
	pdfsubject = {CS461 Problem Statement},
	pdfpagemode = UseNone
}

\begin{document}
	\title{\huge Problem Statement \\ CS 461 Fall 2016}
	\author{\large \name}
	
	\maketitle
		\begin{abstract}The purpose of this project is to produce an iOS/Android application for Calvary Chapel of Corvallis that will allow members to access a plethora of information all in one localized space. 
		The Church's current website does not provide an interface where current members of the church can very quickly access important information such as events, bulletins, and messages from the service. 
		The issue is that the church believes that people do not have the time or the knowhow to open up the website to access the information they need. 
		Also printing out the bullatins every week can be expensive and ineffective.
		The desired application will be simple enough for anyone to use while providing back end access for staff to easily upload new information to the app. 
		The priorities lie in maximizing the usability of the app and providing bulletin, schedule, video, and donation functionality.  
		\end{abstract}
	
	\clearpage	
		
	\section*{Problem Definition}
	Calvary Chapel Corvallis is a congregation of many types of people.
	Some of the members are more technical, while some have very little experience with technology. 
	Currently, the main form of communication with its members, is a website that provides a copious amount of information about the church. 
	The website is somewhat simple, in that it is relatively easy to find the information one needs, even if they aren't as accustomed to using technology. 
	However, to find this information, one may need to perform several clicks. 
	Additionally, not all of the information is relevant to current church members. 
	Some of the website is mainly focused on introducing new members to the ongoings of the church, while some of it is geared towards the current members.
	The main problem with using the website as the only form of communication is that there are multiple pieces of essential data found on many different pages on the website.
	While this might be helpful for members who are looking to join the church and want to study the website, it is not helpful for current members who want to see just the necessary material on a single home page. 
	What would benefit the people using the website is a centralized location that has a main page with links to all of the critical sections that current members of the church would be interested in quickly accessing.
	
	What is needed is a simple application that enables members to listen to recent messages, view upcoming events and sign up to volunteer, read the current e-bulletin, and quickly donate to the church. 
	Additionally, the employees of the church need to be able to easily update the calendar, upload new bulletins and messages, and apply these changes to the application using their current management software. 
	There is the possibility that the application could include the ability to display the live-stream to the members who are unable to be physically present. 
	Finally, members need access to the Bible, which should be integrated into the app, so that they can follow along during the service. 
	We think that this is a worthwile idea because the website simply lacks a lot of the above functionality.
	This project is not so much the creation of a replacement for the website, but a seperate entity that works alongside the website as the next step for the members.
	The website should introduce people to the church, the app should be what they go to after they decide that Calvary is their home.
	
	\section*{Problem Solution}
	Our team aims to create an iOS/Android app that will provide the most pertinent information for the churches core of membership. 
	The application will have many features including an E-bulletin, the church schedule, the ability to watch past sermons, and the ability to read the bible from within the app. 
	All of this will be wrapped up into a highly usable product that even the least tech savvy individuals will be able to easily navigate. 
	The app will draw its data from the churches database for minimal outside maintenance. 
	We have the stretch goals of including live streaming services to the app along with a map of the church campus for members to locate different people/events.  
	We will work with the clients existing web development team to make sure the app is polished in a way that matches the style of their website. 
	This should meet the churches desires for a mobile app that provides the desired functionality for their members.

	At the expo we should have a board showing the different functionality of the app and the process of how we went about creating it. 
	However the big draw should be people ability to download the app themselves and navigate it. 
	At the very least they should be able to demo it preinstalled on one of our test devices.
	
	\section*{Performance Metrics}
	We will measure the success of the project primarily on the usability of the application. 
	If the church body cannot easily navigate the app and find the desired info, it will be less useful then just opening the website on their phone?s browser. 
	Secondary to usability will be the ability for the user to access the bulletin, the church schedule, and watch/listen to past messages. 
	Finally the next metric we would like to meet after the above two is to have the app update the information on it by pulling info from their existing management software.
	
	\clearpage
	
	\section*{Signatures}
	
	\vspace{1in}
	\signature{Client}
	\vspace{1in}
	\signature{Courtney Bonn}
	\vspace{1in}
	\signature{Maxwell Dimm}
	\vspace{1in}
	\signature{Kevin Stine}
	
		
	
 
\end{document}
