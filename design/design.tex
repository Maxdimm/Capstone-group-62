\documentclass[letterpaper,10pt,draftclsnofoot,onecolumn,titlepage]{IEEEtran}

\usepackage{graphicx}
\usepackage{amssymb}
\usepackage{amsmath}
\usepackage{amsthm}
\usepackage{alltt}
\usepackage{float}
\usepackage{color}
\usepackage{url}
\usepackage{enumitem}
\usepackage{pstricks, pst-node}
\usepackage{geometry}
\usepackage{array}


\geometry{margin = .75in}

\usepackage{hyperref}

\newcommand*{\signature}[1]{%
	\par\noindent\makebox[3.5in]{\hrulefill} \hfill\makebox[3.0in]{\hrulefill}%
	\par\noindent\makebox[3.5in][l]{#1}	    \hfill\makebox[3.0in][l]{Date}%
}%

\def\name{Kevin Stine, Courtney Bonn, Maxwell Dimm}
\def\team{Calvary Chapel Corvallis}
\def\grp{Group \#62}

\hypersetup{
	colorlinks = true,
	urlcolor = black,
	linkcolor = black,
	pdfauthor = {\name},
	pdftitle = {CS461 Design Document},
	pdfsubject = {CS461 Design Document},
	pdfpagemode = UseNone
}

\begin{document}
	\title{\huge \team \\ Design Document \\ CS 461 Fall 2016}
	\author{\large \name \\ \grp}

	

	\maketitle

	
		\begin{abstract}The purpose of this project is to produce an iOS/Android application for Calvary Chapel of Corvallis that will allow members to access a plethora of information all in one localized space.
		The Church's current website does not provide an interface where current members of the church can very quickly access important information such as events, bulletins, and messages from the service.
		The desired application will be simple enough for anyone to use while providing back end access for staff to easily upload new information to the app.
		The priorities lie in maximizing the usability of the app and providing bulletin, schedule, video, and giving functionality.
		We will work with the existing Calvary Chapel web development team to create a product that is seamlessly integrated with their already existing network.
		\end{abstract}


	\clearpage

	\tableofcontents

	\clearpage
	
	\section{Overview}
	
		\subsection{Scope}
			We will be implementing a mobile application for both iOS and Android platforms. 
			This app will provide the members of Calvary Chapel Corvallis a centralized hub that will allow them to access the most important information about the church. 
			Users will be able to view announcements, a calendar, e-bulletins, and sermons. 
			Users will also be able to donate to the church. 
			Three people will be involved in the implementation of this software and it will be done during October 2016 and June 2017. 
		
		\subsection{Purpose}
			The purpose of this design document is to detail how the application software will be designed. 
			We will discuss how we will meet the requirements for our church application and discuss the structure of the application. 
		\subsection{Intended Audience}
			The intended audience of this design document will be our clients, the teachers of CS 461, as well as the teacher's assistants. 
	
		
	\section{Definitions}
	
	\section{Conceptual model for software design descriptions}
	
		\subsection{Software design in context}
		
		\subsection{Software design descriptions within the life cycle}
		
			\subsubsection{Influences on SDD preparation}
				The key influence on this software design document is the software requirements specification document (SRS) that has previously been documented. 
				The requirements listed in the SRS will greatly determine the design of the software and how we implement this project. 
			
			\subsubsection{Influences on software life cycle products}
				This software design document may lead to necessary changes in the SRS. 
				Throughout development, it's possible that there will be design changes that require us to change details in the SRS. 
				Testing may also be changed based on the design document. 
				
			\subsubsection{Design verification and design role in validation}
				In order to determine if the application has met requirements, we will perform user testing. 
				This will involve having users try to use the software in the intended manner and see if they are successful. 
				Success will be determined by how many users are successfully able to use the application without errors or issues. 
	
	\section{Design description information content}
	
		\subsection{Introduction}
			Within this design document, there are many required contents. 
			We will identify the software design document and it's stakeholders. 
			We will discuss design concerns and selected design viewpoints. 
			We will also discuss design views, design overlays, and the design rationale. 
		
		\subsection{SDD Identification}
			A valid software design document includes the following parts: 
			\begin{itemize}
				\item{Date of issue and status}
				\item{Scope}
				\item{Issuing organization}
				\item{Authorship}
				\item{References}
				\item{Context}
				\item{One or more design languages for each design viewpoint used}
				\item{Body}
				\item{Summary}
				\item{Glossary}
				\item{Change history}
			\end{itemize}
	
		\subsection{Design stakeholders and their concerns}
			The design stakeholders of the Calvary Chapel of Corvallis application are the developers of the app and the team at the church. 
			Design concerns of the stakeholders include creating an application that is user-friendly and very simplistic in design features.  
			The final application will be designed in a way that will ease this concern, as the intended design will be a very easy-to-use application. 
		
		\subsection{Design views}
		
		\subsection{Design viewpoints}
		
		\subsection{Design rationale}
		
		\subsection{Design languages}
		
	\section{Design Viewpoints}
	
		\subsection{Introduction}
		
		\subsection{Context Viewpoint}
		
		\subsection{Composition Viewpoint}
		
		\subsection{Logical Viewpoint}
		
		\subsection{Dependency Viewpoint}
		
		\subsection{State Dynamics Viewpoint}
		
		\subsection{Interaction Viewpoint}
		
	\section{Conclusion}
	
	

\end{document}