\documentclass[letterpaper,10pt,draftclsnofoot,onecolumn,titlepage]{IEEEtran}

\usepackage{graphicx}
\usepackage{amssymb}
\usepackage{amsmath}
\usepackage{amsthm}
\usepackage{alltt}
\usepackage{float}
\usepackage{color}
\usepackage{url}
\usepackage{enumitem}
\usepackage{pstricks, pst-node}
\usepackage{geometry}
\usepackage{array}
\usepackage{listings}
\usepackage{caption}
\usepackage{subcaption}
\usepackage{import}
\usepackage[final]{pdfpages}

\renewcommand{\lstlistingname}{Code}

\geometry{margin = .75in}

\usepackage{hyperref}

\usepackage[acronym]{glossaries}

\makeglossaries

\newglossaryentry{iOS}{name={iOS}, description={A mobile operating system created and developed by Apple Inc. exclusively for Apple's hardware}}
\newglossaryentry{ModelVC}{name={Model-View-Controller}, description={A design pattern that assigns objects in an application one of three roles: model, view, or controller. Also called MVC}}
\newglossaryentry{Android}{name={Android}, description={A mobile operating system developed by Google, based on the Linux Kernel and designed primarily for touchscreen mobile devices}}
\newglossaryentry{App}{name={app}, description={A software application designed to run on mobile devices such as smartphones or tablet computers}}
\newacronym{ccb}{CCB}{Church Community Builder}
\newacronym{sdd}{SDD}{Software Design Document}
\newacronym{srs}{SRS}{Software Requirements Specification}
\newacronym{uml}{UML}{Unified Model Language}
\newacronym{mvc}{MVC}{Model-View-Controller}
\newacronym{xml}{XML}{EXtensible Markup Language}
\newacronym{ui}{UI}{User Interface}




\graphicspath{{figures/}{pictures/}{images/}{./}}


\newcommand*{\signature}[1]{%
	\par\noindent\makebox[3.5in]{\hrulefill} \hfill\makebox[3.0in]{\hrulefill}%
	\par\noindent\makebox[3.5in][l]{#1}	    \hfill\makebox[3.0in][l]{Date}%
}%

\def\name{Kevin Stine, Courtney Bonn, Maxwell Dimm}
\def\team{Calvary Chapel Corvallis}
\def\grp{Group \#62}

\hypersetup{
	colorlinks = true,
	urlcolor = black,
	linkcolor = black,
	pdfauthor = {\name},
	pdftitle = {CS463 Final Report},
	pdfsubject = {CS463 Final Report},
	pdfpagemode = UseNone
}

\begin{document}
	\title{\huge \team \\ Final Report\\ CS 463 Spring 2017}
	\author{\large \name \\ \grp}



	\maketitle

		\begin{abstract}The purpose of this project is to produce an iOS/Android application for Calvary Chapel of Corvallis that will allow members to access a plethora of information all in one localized space.
		The Church's current website does not provide an interface where current members of the church can very quickly access important information such as events, bulletins, and messages from the service.
		The desired application will be simple enough for anyone to use while providing back end access for staff to easily upload new information to the app.
		The priorities lie in maximizing the usability of the app and providing bulletin, schedule, video, and giving functionality.
		We will work with the existing Calvary Chapel web development team to create a product that is seamlessly integrated with their already existing network.
		\end{abstract}

		\clearpage

		\tableofcontents

		\clearpage

\section{Introduction}

Our project centered around a local church in Corvallis, Oregon, Calvary Chapel of Corvallis.
The church requested our help in creating a mobile application that would work together with their existing website and be used by the members of the congregation.
Our main goal was to create an application that was compatible with both iOS and Android smart phones, had a simple design and functionality so all people could use it easily, and incorporated the most important pieces of their current website.
It was decided that the main sections the church wanted to see on the applications were the bulletins, events, the ability to donate, and the most recent message video.

The client team was led by project manager, Desiree Gorham.
She acted as a channel of communication between us and the senior staff at the church to make sure we were designing the app in the way they had intended.
The church was not involved in development, but did offer advice and ideas as to how they wanted the final product to look and function.

The members of our team were Courtney Bonn, Maxwell Dimm, and Kevin Stine.
The project requirements were spread among us, though we did change those assignments throughout the process as well as helped each other when needed.
The roles were as follows:

\begin{itemize}
	\item Courtney Bonn: iOS - Bulletin Page; Android - Bulletin, Donation, Events Page
	\item Max Dimm: iOS - Donation, Messages Page; Android - Messages Page
	\item Kevin Stine: iOS - Events Page
\end{itemize}

\section{Original Requirements}

	Please see the following page for our requirements documentation. 

	\includepdf[pages={2-8}]{originals/requirements.pdf}

\section{Requirement Changes}
	Due to having well established our client'��s desires early and maintaining heavy communication fall and early winter term, we were able to make little to no changes to our requirements doc. 
We made sure to establish exactly what our app should do and look like early on and work towards that same goal throughout our development. 
Our client towards the end of winter term and in spring term requested a few altercations to the design that we were able to complete such as adding additional navigation to the events page. 
The only requirement we did not add was the addition of push notifications due to how late in the development process it was requested. 
However, due to the nature of the changes, they still fit within the established requirement document and hence did not need changes.

	Upon presenting the finished product to our client we got feedback on the overall feel and design of the app. 
While it fit all of our originally established requirements there were still a few modifications our client requested. 
We added a date picker to the events, changed the design of the homepage, and added loading wheels to the pages for added usability. 
Our client requested that we add support for multiple languages, but due to timing, we were unable to deliver on that request.

	Our final Gantt Chart documenting the timeline of our project can be seen on the following page. 
	
	\includepdf[pages={-}]{originals/GanttChart.pdf} 

\section{Original Design Document}

	Please see the following page for our design documentation. 

	\includepdf[pages={2-12}]{originals/design.pdf}

	\subsection{Design Changes}
	
	Because our original design plan was a very simplistic idea for the applications, we did not end up changing any of our design decisions. 
	The client wanted a clean looking app that looked similar to their website. 
	The website has a white background and dark gray headings.
	The only color on the website is the church's logo, as well as various pictures and logos. 
	Our application directly reflects the website with the same color scheme and simple, clean design. 

\section{Original Technology Review}

	Please see the following page for our technology documentation.
	
	\includepdf[pages={2-23}]{originals/tech-review.pdf}

	\subsection{Technology Changes}
	Throughout the process of building our applications, the church was also busy working on their own website.
	The new church website used Wordpress as the software backing the entire site.
	Because we wanted to limit the amount of extra work required to maintain the application, it made sense to change the way we were incorporating the bulletin page into the app.

	Originally, we proposed to incorporate the bulletin page using the database, \gls{ccb}.
	The bulletin announcements were going to be handled in this database, which we could then request through the API.
	However, with the new website, the bulletins were no longer going to be updated on the database.
	This meant, should we choose to use this database, the church would have to update the database as well as the website in order to update the app.

	Because their new website used Wordpress, we were able to make use of the Wordpress REST API.
	By utilizing this API, we could request the content on the bulletin page and using a JSON parser, we could display the results on the application.
	Now, whenever the website updates, the app will update as well.

	There were no other technology changes.

\section{Weekly Blog Posts}

	\import{./}{blogposts}

\section{Final Poster}

	Please see the following page for our poster that was presented at Engineering Expo 2017.

\includepdf[pages={-}]{originals/poster.pdf}

\section{Project Documentation}

Our project works in two different ways, depending on if it is being ran as a developer or as a consumer.
Because the end result is an application, available on both iOS and Android platforms, the project's structure is very simple.
There are two sets of project files, one for each platform, and within the files the code that can be executed in order to run the application is available.

As a consumer, one can download the application from the app store once it has been published.
After downloading, it will be available to use on the corresponding phone.
The app itself takes up less than 40 MB of space on a phone.

As a developer, the applications can be ran on a computer using two different softwares: Xcode and Android Studio.
These softwares are required in order to run the code.
They can be downloaded from \url{https://developer.apple.com/xcode/} or \url{https://developer.android.com/studio/index.html}.
One caveat, though, is that to be eligible to download Xcode, one must be on a Mac computer.
Android Studio can be downloaded from both Mac and Windows computers.

Once the required software is installed, the code can be downloaded from \url{https://github.com/ikaikastine/capstone-group-62} with the code living in the folders iOS and Android.
The projects can then be opened in the corresponding software and ran accordingly.
The software allows one to run the code on either a built in simulator or an actual device if it is connected to the computer.



\section{Learning New Technology}
Because mobile development was new to all of us, we had to spend a good portion of our time researching and becoming accustomed to using the mobile development software, Xcode and Android Studio.
The main sources which we used to learn these platforms, were produced by Apple and Android Studio.
Apple had a website in which it walked us through how to begin developing an iOS App on Xcode.
It taught us how to develop in Swift as well as how to use the program \cite{AppleSwift}.
Android Studio had a similar website, as well as sample apps we could walk through designing \cite{AndroidStudio}.

Besides learning the main platforms, we had to do a copious amount of research on how to implement each part of the app.
A website that was instrumental in learning how to parse the Wordpress pages into JSON objects and display the JSON response was Grok Swift \cite{JSONSwift}. Because we also needed to know how to handle JSON in Android, a thread on Stackoverflow was very helpful \cite{JSONAndroid}.

A tutorial on \url{www.androidbegin.com} was used to understand how to handle XML parsing and displaying the data on an Android app \cite{XMLAndroid}.
We needed this information to handle the events calendar through \gls{ccb}.

For the iOS Events page, a lot of time was spent looking at the Apple Developer API reference at \url{developer.apple.com/reference} to get familiar with the references for various portions of the application.
For the Events page specifically there was a lot of time spent looking at the DateFormatter reference, the NSDate reference, and UIDatePicker reference.
In addition to utilizing the developer API reference, Stack Overflow at \url{stackoverflow.com} was pretty critical in helping with debugging the code.
I found a lot of discrepancies between Swift 3.0 and older versions, so when my code had a bug, it was tough to find specific help related to that version of swift.
However I think that stackoverflow was instrumental in my success in debugging and with adding certain features like the UIDatePicker that utilized a particular type of object with the toolbar.

While there were other websites that we searched and read, the above websites were the most helpful in learning the new technology.
As far as print documentation, we did not utilize any reference books in our learning.
Our assigned teacher's assistant, Vedanth, did provide some suggestions to help us with our learning, as he had some experience in mobile development from his own senior project.

\section{Team Reflection}

	\subsection{Courtney Bonn}

		Because I had not worked with mobile development before I decided to choose this project, I was given the opportunity to learn an abundance of technical information.
		I went into this project with little understanding as to how an app worked and how to create my own.
		Fast forward to the end of the year and I know feel comfortable developing in both iOS and Android platforms, a skillset I am excited to get to add to my resume.
		Specifically, I learned how to code in Swift and brushed up on my Java language skills.
		I learned two new platforms to code in, Xcode and Android Studio.
		I also learned how to work with a REST API and JSON objects, as well as XML parsing and responses.
		Finally, I learned the whole process of creating an app from start to finish.

		The main non-technical bit of information that I learned this year was how to produce accurate and helpful documentation for a large project.
		While documentation can feel time-consuming, it ends up saving a lot of time because there is already a roadmap for the project.
		Beginning development with no documentation would have felt overwhelming and chaotic and more likely than not would have caused development to take at least twice the amount of time.
		Knowing how to produce the necessary documentation is something I am very grateful to have learned in this course.

		Until this class, most assignments I had worked on were weekly assignments, with a few term projects here and there.
		Even then, term projects did not usually take up the entire term, but perhaps the last couple of weeks.
		I have never experienced working on the same project for a year before.
		Learning to keep focus and interest in a project for this amount of time, was a valuable skill that I am taking away from this project.


		The biggest lesson I learned about project management, was that it is very easy to fall behind in a project if you go even one week without working on it.
		There were times where I could not work on the project for whatever reason, and then I felt I was behind in where I should be in project progression.
		The lesson I will take away is that even putting an hour into the project every week or a few minutes a day will help keep the project going smoothly.

		In my college career, I have worked in many group projects.
		More often than not, these projects lasted a few weeks to a term at most, and then we went our separate ways.
		To work with the same group for almost an entire year was a new experience for me and one that I enjoyed.
		I learned that you have to help one another because you have to work as a time--you cannot leave anyone behind.
		I also learned that you have to be patient with each other because not everyone is going to be able to commit the same amount of time or effort at all times.
		Because we all had different classes to focus on as well, we had to work together to make sure our project was not falling behind if we needed to work on other classes.


	\subsection{Max Dimm}
	The primary motivator behind requesting this project was who we would be working for. 
As someone who attends church regularly and finds value in helping out a church in need, I figured that if I was to be working on a project for a few months that it would be best if I was doing it for an organization I found value in. 
Secondly, I had never worked on any form of app development before this and it was something that I had wanted to do for a while. 
I think learning the fundamentals of building an app is a skill that will come invaluable later on in life and will pay dividends later. 
I had no idea what language it would be coded in or what tools we would need to use, but that mattered less to me.

	While working on our papers fall term, I learned a lot about the value of version control and how to properly make use of Github. 
Before I had used Github to host work for assignments, but I was not using it as it was intended. 
I would just do all the work locally and copy the files over to Github so that I could turn it in on that platform which offered next to zero version control. 
However, slowly over the term I figured out how to use Github and other software that works with it properly. 
This allowed us to not work on the same assignment twice and stay productive without wasting effort. 
Also, we made use of a software called slack to stay in contact with one another and keep us updated on Github commits. 

	I think one of the major benefits of this class is getting work like experience. 
This project and class as a whole was built to look and feel like a real work assignment with all the same requirements and deadlines. 
We worked in a team and had a client we answered directly to. I, having not done an internship before, find a lot of value in this work experience along with learning new skills I had not explored before. 

	One of the big things I learned this term was how to diagnose issues when dealing in a subject or language I had not worked in before. 
I found myself a couple times running into issues that I did not understand without having many resources to go to. 
I had to try to locate the root of the issue and find similar issues online and how they went about solving them. 
However the locating of the primary issue was often the hardest part. If you do not know what you are dealing with, it will be hard to realize what the problem is.
On a more specific level, I was able to learn a ton about working in both android studio and Xcode. 
I had no experience with swift before and minimal with java. 
Now I feel I could work on either platforms without too much confusion.

	Working in a team was a great experience. 
I learned a lot about the evolution of projects over the long term. 
It is rare to work on something with others for so long in school so this was unique in that way. 
I think the key to our success was strong communication and setting out goals for one another. 
I found that being upfront about what we were doing and what we needed went a long way to help us help each other. 
We were not constantly working on the project due to life happening around us, so we had to have grace in different areas which was nice. 
I know I went through rough patches and having partners who understood was a blessing. 

	If I was to do this all over again I would start looking up swift and android studio tutorials earlier. 
I felt I spent too much time winter term trying to learn what it was that I was doing, before I could actually go about implementing anything. 
I spent a long time trying to diagnose problems which now feels like wasted time, so finding a resource that could answer questions on the subject would have been super helpful. 
Also, I think it would have been beneficial to work on the IOS and android app alongside each other instead of one first then the other later. 
A lot of the work we were doing transferred over fairly smoothly so it would have been beneficial to try to do both at the same time. 
Also, in this project I felt like I took a backseat in a lot of areas. 
If I was to do this again I would try to be more proactive in taking on different assignments or mini-projects. 



	\subsection{Kevin Stine}
		There were a few reasons why I choose to partake in this project for senior design.
		Firstly, I really wanted to get some hands on experience with mobile development as that was an area of programming and design I had always wanted to explore, but had not really had formal instruction on how to do so.
		I decided that this could be a great opportunity to learn the fundamentals and basics of mobile development to add to my portfolio.
		Second, this project intrigued me due to the fact that Calvary Corvallis was using Church Community Builder, and the local church plant that I am a part of also uses that particular backend software.
		I thought this would be the perfect opportunity to learn how make a mobile app that integrates with CCB in the hopes that one day I could create a similar type of application for my church.
		This project was a great opportunity for me to learn more about mobile development, and really learn the intricacies of Swift as well as some Java.

		I learned a lot of technical information through this project.
		I learned how to use Xcode more in depth as well as Android Studio as the platforms in which we developed for.
		Surprisingly I spent a lot of time with Xcode just learning how to use the inspector to change the various attributes of variables and of elements on the storyboard.
		While I did not end up doing too much Android development, I did still learn how to use Android Studio, learned how to utilize the xml files that make up and Android project, and learned how to setup an Android device to use for debugging and demoing purposes.
		I learned a ton on Swift as I mainly focused on iOS, and really feel like I have a solid foundation now for being able to create my own applications.
		I learned how to utilize the storyboard, inspector, created a tabbed application, utilize the NSDate and DateFormatter references and create a UIDatePicker with a toolbar.
		While I obviously learned more than just those few things, I think those are the most notable technical things I learned how to use.
		In addition I was able to continue to enhance and refine my skills at using git as it was something we utilized throughout the entire process of the project.

		On the non-technical side, I learned a lot about how to write good documentation and how to take the information, needs and ideas from our client and formulate that into specific requirements that we could use as benchmarks.
		With Fall term being mostly centered around writing documents, I found that it also helped enhance my writing skills while improve on my ability to analyze and think through specific processes and ideas.
		In addition I was able to learn a lot about time management through this project.
		Having deadlines for documents really helped me set easy goals, however once we got to Winter term where deadlines were more fluid, it really made me have to manage my time well and make sure that I was staying on track to reach our goals.
		I learned that when it comes to project work, it is really important to make sure you set your requirements and make sure that you have goals to reach throughout the entirety of the project.

		In terms of project management and working with a team, I learned that communication is key to the success of the project.
		Utilizing tools such as Slack really helped boost our communication as a team, allowing us to make sure we were all contributing and striving towards the same goals.
		It also made sharing the work easier as we could just ask for help debugging or to see if someone else could take a look at the code to see if they could figure out what was going wrong.
		I also found that working with a team really requires everyone to put in their best effort, while we all have different things going on outside of school, it is important to keep in mind that we are on the same team and have the same goals.
		It also really helps to designate who will do what parts of the project so everyone has a clear idea of what is required of them.
		Knowing that we each have a part to play in the whole project really helped us moving forward as we knew that once we all finished our individual pieces, it would come together to make a nice application.

		If I could do it all over again I would definitely do some more research on the front end.
		While I had a general idea of how to use Xcode and a basic idea of what the format for an iOS app was, I did not really have a good understanding of how it all fit together.
		While doing research I learned about the Model View Controller and it really helped me understand more of what was going on behind the scenes.
		I think if I had known more technical aspects of a mobile app going in, I would have been able to better understand how the various pages linked together.
		I also would have started development sooner as it seemed we kind of had to rush towards the end to make sure we fixed all of our bugs before the deadline.
		While we did do development throughout all of Winter term, I think if we had gotten the basics and easy stuff out of the way first, rather than researching how to implement the whole XML Parser in my case, things would have gone smoother.
		I also would have broken down my section a bit more and mapped out how I was going to approach development.
		I basically started development by going after the whole XML Parser without realizing that I would need to connect to the CCB API in order to access that information and would need to create some sort of date to keep track of whatever the current date was.
		I think if I had started with the Date portion I would have been able to implement the XML Parser a bit easier. 

	\bibliographystyle{IEEEtran}
	\bibliography{IEEEabrv,finalreport}

	\appendices

	\section{Essential Code Listings}

	\begin{lstlisting}[caption=iOS EventViewController Snippet]
func createDatePicker() {
    // format the picker
    datePicker.datePickerMode = UIDatePickerMode.date
    //toolbar
    let toolbar = UIToolbar()
    toolbar.sizeToFit()
    // bar button item
    let doneButton = UIBarButtonItem(barButtonSystemItem:
			.done, target: nil, action: #selector(donePressed))
    toolbar.setItems([doneButton], animated: false)
    changeDate.inputAccessoryView = toolbar
    changeDate.inputView = datePicker
}
\end{lstlisting}

This code creates the datePicker which allows the user to select the month, day and year.

\begin{lstlisting}[caption=iOS EventViewController Snippet]
func donePressed() {
    // format date
    let dateFormatter = DateFormatter()
    dateFormatter.dateFormat = "yyyy-MM-dd"
    pickerTracker = true
    startDate = dateFormatter.string(from: datePicker.date)
    updateTable()
    self.view.endEditing(true)
}
\end{lstlisting}

This code is the selected action for the DatePicker.

\begin{lstlisting}[caption=Android XML Parser]
 public String getXmlFromUrl(String url) {
       OkHttpClient client = new OkHttpClient();
           final String basic = "Basic " + Base64.encodeToString(CREDENTIALS.getBytes(),
           Base64.NO_WRAP);
           String str = null;
           Request request = new Request.Builder()
                   .url(url)
                   .header("Authorization", basic)
                   .build();

           try {
               Response response = client.newCall(request).execute();
               str = response.body().string();
           } catch (IOException e) {
               e.printStackTrace();
           }

       return str;
   }
\end{lstlisting}

\begin{lstlisting}[caption=iOS JSON Response]

{
  "id": 1038,
  "date": "2016-10-27T19:22:53",
  "date_gmt": "2016-10-27T19:22:53",
  "guid": {
    "rendered": "http://www.calvarycorvallis.org/?page_id=1038"
  },
  "modified": "2017-03-18T11:48:50",
  "modified_gmt": "2017-03-18T18:48:50",
  "slug": "bulletin",
  "status": "publish",
  "type": "page",
  "link": "https://www.calvarycorvallis.org/bulletin/",
  "title": {
    "rendered": "This Week&#8217;s Bulletin"
  },
  "content": {
    "rendered": "<p>all bulletin content would be here...
    ...
    },
    Additional, unrelated JSON returned below here...
 }

\end{lstlisting}

\begin{lstlisting}[caption=iOS JSON Parser]
 do {
 	guard let bulletin = try JSONSerialization.jsonObject(with: responseData,
	options: []) as? [String: AnyObject] else {
	         print("error trying to convert data to JSON")
                 return
         }

        	guard let bulletinContent = bulletin["content"]?["rendered"] as?
	String else {
                  print("Could not get bulletin content from JSON")
                  return
         }
         let actualContent = bulletinContent.replacingOccurrences(of: "<[^>]*.", with:
         "", options: .regularExpression, range: nil)

         DispatchQueue.main.async{
 	          self.jsontext.text = actualContent
         }
} catch  {
         print("error trying to convert data to JSON")
         return
}
\end{lstlisting}

\begin{lstlisting}[caption=Android JSON Parser]
if (response != null) {
	try {
		JSONObject jsonResponse = response.getJSONObject(TAG_CONTENT);
                 String jsonData = jsonResponse.getString(TAG_RENDERED);
		 textView.setText(jsonData);
                  Log.e("App", "Success: " + response.getString("yourJsonElement"));
	} catch (JSONException ex) {
                    Log.e("App", "Failure", ex);
        }
}
\end{lstlisting}

\begin{lstlisting}[caption=iOS Load into WebView]
let htmlCode = "<!DOCTYPE HTML><html><head><style> body {color: #5b5e5e; font-family:
'Lora', Palatino;} a { border-bottom: 1px solid #fbaf17; color: #fbaf17;
text-decoration: none; }
.staff a { border-bottom: 0px none; } a:focus, a:hover { border-bottom: 1px solid #fbaf17;
color: #b17b0e; }</style></head><body>" + bulletinContent + "</body></html>"

self.bulletinWeb.loadHTMLString(htmlCode, baseURL: nil)
		\end{lstlisting}

\begin{lstlisting}[caption=Android Load into WebView]
String bulletinContent = "<!DOCTYPE HTML><html><head><style> body {color: #5b5e5e;
font-family: 'Lora', Palatino;} a { border-bottom: 1px solid #fbaf17; color: #fbaf17;
text-decoration: none; } .staff a { border-bottom: 0px none; } a:focus, a:hover {
border-bottom: 1px solid #fbaf17; color: #b17b0e; }</style></head><body>" +
jsonData + "</body></html>";

myWebView.loadDataWithBaseURL("file:///android_asset/", bulletinContent,
"text/html", "utf-8", null);
myWebView.getSettings().setAllowFileAccess(true);
\end{lstlisting}


\begin{lstlisting}[caption=iOS Donation Page]
 	let donateURL = URL (string: "https://www.calvarycorvallis.org/give/")
        let requestObj = URLRequest(url: donateURL!)
        donateView.loadRequest(requestObj)
        donateView.delegate = self
        donateView.scrollView.delegate = self
        donateView.scrollView.isScrollEnabled = false
\end{lstlisting}

\begin{lstlisting}[caption=Android Donation Page]
@Override
    public void onPageFinished(WebView view, String url) {
            myWebView.loadUrl("javascript:(function() { " +
            "document.getElementsByClassName('site-header')[0].style.display='none'; " +
            "document.getElementsByClassName('footer-widgets')[0].style.display='none; " +
            "document.getElementsByClassName(content')[0].style.display='none'; " + "})()");
            myWebView.setVisibility(View.VISIBLE);
            myWebView.getSettings().setLoadWithOverviewMode(true);
            myWebView.getSettings().setUseWideViewPort(true);
   	}
   });
        myWebView.setVisibility(View.GONE);
        myWebView.loadUrl("https://www.calvarycorvallis.org/give/");
\end{lstlisting}

\begin{lstlisting}[caption=Android Messages Page]
public View onCreateView(LayoutInflater inflater, ViewGroup container,
Bundle savedInstanceState) {
        myView = inflater.inflate(R.layout.fourth_layout, container, false);

        String videoLink = "<html><iframe id=\"ls_embed_1493363421\" src=\"
        https://livestream.com/accounts/18343788/events/7279945/videos/154327352/player
        ?width=960&height=540&enableInfo=false&defaultDrawer=&autoPlay=true&
        mute=false\" width=\"960\" height=\"540\" frameborder=\"0\" scrolling=\"no\"
        allowfullscreen> </iframe></html>";

        myWebView = (WebView) myView.findViewById(R.id.messagesView);

        WebSettings webSettings = myWebView.getSettings();
        webSettings.setJavaScriptEnabled(true);
        myWebView.getSettings().setLoadWithOverviewMode(true);
        myWebView.getSettings().setUseWideViewPort(true);
        myWebView.getSettings().setBuiltInZoomControls(true);
        myWebView.loadData(videoLink, "text/html", "utf-8");

        return myView;
		\end{lstlisting}


	\section{Photos}

	\begin{figure}[H]
			\centering
			\begin{subfigure}{.5\textwidth}
 				 \centering
  				 \includegraphics[width=.4\linewidth]{androidevents}
 				 \caption{A List of Events}
  				 \label{fig:sub1}
			\end{subfigure}%
			\begin{subfigure}{.5\textwidth}
		         	\centering
 				 \includegraphics[width=.4\linewidth]{androiddetails}
 				 \caption{An Event Details}
 				 \label{fig:sub2}
			\end{subfigure}
			\caption{Android Event Page}
			\label{fig:event}
		\end{figure}


		\begin{figure}[H]
			\centering
			\begin{subfigure}{.5\textwidth}
 				 \centering
  				 \includegraphics[width=.4\linewidth]{androidbulletin}
 				 \caption{Android}
  				 \label{fig:sub1}
			\end{subfigure}%
			\begin{subfigure}{.5\textwidth}
		         	\centering
 				 \includegraphics[width=.4\linewidth]{iosbulletin}
 				 \caption{iOS}
 				 \label{fig:sub2}
			\end{subfigure}
			\caption{Bulletin Page}
			\label{fig:bulletin}
		\end{figure}

		\begin{figure}[H]
			\centering
			\begin{subfigure}{.5\textwidth}
 				 \centering
  				 \includegraphics[width=.4\linewidth]{androiddonate}
 				 \caption{Android}
  				 \label{fig:sub1}
			\end{subfigure}%
			\begin{subfigure}{.5\textwidth}
		         	\centering
 				 \includegraphics[width=.4\linewidth]{iosdonate}
 				 \caption{iOS}
 				 \label{fig:sub2}
			\end{subfigure}
			\caption{Donation Page}
			\label{fig:donation}
		\end{figure}

	\begin{figure}[H]
			\centering
			\begin{subfigure}{.5\textwidth}
 				 \centering
  				 \includegraphics[width=.4\linewidth]{androidmessages}
 				 \caption{Android}
  				 \label{fig:sub1}
			\end{subfigure}%
			\begin{subfigure}{.5\textwidth}
		         	\centering
 				 \includegraphics[width=.4\linewidth]{iosmessages}
 				 \caption{iOS}
 				 \label{fig:sub2}
			\end{subfigure}
			\caption{Messages Page}
			\label{fig:message}
		\end{figure}


\end{document}
