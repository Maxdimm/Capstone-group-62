
\subsection{Fall 2016}
	
		\subsubsection{Week 3}
		
			\paragraph{Courtney Bonn}
			This week we met with our client on Tuesday and met the rest of the staff at Calvary Church. At the meeting we talked about the basic idea of the application, as well as some bigger goals that we want to strive for. We also discussed what kind of design they were looking for. Like the functionality of the app, they are looking for a relatively simple design.

This upcoming week we are planning to meet with the church's website designer/developer. He has information about the church's management software that will assist us in integrating the calendar into the application. Personally, I plan to start doing a lot of research this upcoming week on creating iOS/Android applications.

			\paragraph{Max Dimm}
			This week we actually went to calvary church to meet with our client and her churches existing dev team. We went over what their expectations and desires were out of the project. A lot of this info can be seen on our problem statement. However the just of it was that they wanted an easy to use app for IOS and android that gives their existing members an easy way to access weekly information. The info they wanted included on the app was things like a bulletin, schedule, videos of past messages, and the bible. There were other small things but those were the major ones.

Also this week we set up this blog and stated/finished our problem statement. For the problem statement I wrote about our solution and what we should be expected at the expo. I wrote around 300-350 words and Courtney helped me out by adding it to the tex document.
			
			\paragraph{Kevin Stine}
			This week we met with our client and hammered out the details of what our project will look like. Since we are building a mobile app based on the mobile version of their website, we met with the media and web guys to get a better idea of what they were envisioning. Since last week we were able to get all the details down and write our problem statement.

This upcoming week I plan on getting more familiar with LaTeX for future write ups, looking into mobile development for iOS and Android, and getting more familiar with the back-end software that the church utilizes.

		\subsubsection{Week 4}
		
			\paragraph{Courtney Bonn}
			We met with our client at the church and we were introduced to their website developer. This was mostly just an introduction where we gathered any additional thoughts he had on the project. He let us know that he would prefer to publish the app at the end of the year, so that way it's under his developer profile. I didn't do as much outside research on app development as I was hoping to do this week, but I am planning on doing much more this upcoming week. We also have the problem statement that we need to edit. I don't think we have too much editing to do on this paper, but I still plan on putting some more time into it. Additionally, we have the requirements document that we are going to start working on. I think the biggest goal this next week is to get a much more solid grasp onto our project, hopefully leading us into a good start on the actual project itself.

			\paragraph{Max Dimm}
			This week we met with our client again to be introduced to their lead web developer. We talked for a bit about the problem statement along with the developers ideas for the project. He was interested about how the end product would be delivered to him and how he can be included in the dev process. He informed us he has the ability to upload apps to the app store. We didn't have too much to do besides this meeting as there were no assigned papers/works alongside this. We will begin the edits on our problem statement when we are emailed our corrected first draft.
			
			\paragraph{Kevin Stine}
			This week we met with our client and got to meet their lead web developer. We picked his brain a bit to get a better idea of what he had in mind for the app, and how we could get access to some of their current resources. We got into talking about the actual deliverable and how we would be handing off the project and whether or not publishing the app would fall on us. This week I also took a look at some of the basics of developing for iOS using Swift. I started going through Apple's Swift Developer Tutorial to get a little more familiar with Xcode and Swift. Next week we'll be going through the requirements document and really hammer out the details of all the requirements we have on this project.
			
		\subsubsection{Week 5}
		
			\paragraph{Courtney Bonn}
			This week we focused on updating our problem statement and drilling down on our requirements document. We reviewed some of the requirements document with our TA to check in and make sure we were headed in the right direction. I did some research on Swift and iOS programming. I downloaded Xcode onto my Mac to start trying and learn Swift. We are at the point where we need to start thinking about whether or not we are going to build two apps (one for iOS, one for Android) or if we are going to build one app that is cross-platform. This upcoming week we need to put in some time researching what might be best for our project. We plan on meeting with our client this week to go over the requirements document and fine-tune it before turning in the final document on Friday 11/4.

			\paragraph{Max Dimm}
			This week we finished up our problem statement and began on our requirements doc. We also met our TA Vee who seemed pretty chill. We decided not to meet with our client this week because we didn't have much to deliver to her at this point, plus we had met every other week up until this point. On my end I was able to help a bit with both of the documents and deliver them on the due dates. I need to be more on top of the documents, because by the time I get involved they most of the time have already taken form. In the future I'd like to help design or help start the documents.
			
			\paragraph{Kevin Stine}
			This week we finished up our problem statement and got that submitted based on what our client approved. I began going through a swift tutorial and have been looking into the basics for creating an application on iOS. This week we will be determining exactly what the client wants in terms of the application. Do they want two apps for both iOS and Android, or would they prefer an application for one platform in particular.
			
		\subsubsection{Week 6}
		
			\paragraph{Courtney Bonn}
			This week we focused on finishing and polishing our requirements document. We got a rough timeline for the rest of our project. We sent our requirements document to get approved by our client, who agreed with everything except for just a few small changes. Next up is beginning to research for the design document and along with that we'll begin working on the technology review. This will consist of us looking into the different technology available for us to use on our project.

			\paragraph{Max Dimm}
			So this week we were pretty much exclusively working on the requirements doc. We were able to get it done but it was a little tense getting our clients attention at one point. Once we were able to get her to work with us we could put the finishing touches on the doc and turn it in. I missed the one class of the week and my first class because I was out of town for the holiday weekend but don't plan on missing much more class. I need to start looking into app development more and finding out how I can contribute to the actual work part of our project from this point on.
			
			\paragraph{Kevin Stine}
			This week our main focus was completing the requirements document. We really took the time to verify that we documented everything that we would need to be doing for this app. This week we'll be getting started with the design document and tech review, so we'll need to really hammer out our design on the application. I plan on looking more into the different design guidelines for both iOS and Android, and coming up with some good designs and ideas for how we want to structure our application. I've been going through a Swift tutorial and will continue to explore swift in the hopes to use it for our iOS application.
			
		\subsubsection{Week 7}
		
			\paragraph{Courtney Bonn}
			We focused heavily on the technology review this week. We didn't meet with our client as we didn't have any new material to go over with her quite yet. We met up just with our group and hashed out the different parts of our system that we need to researched. Once we figured out who was going to be working on each part, we each set out to start our individual research. I took on iOS development platform, iOS user interface organization, and integrating the e-bulletins. I'm fairly confident that each technology I chose will work well with our client, as I kept in mind the requirements and exactly what the client is expecting. This next week we will start working on the design document.

			\paragraph{Max Dimm}
			This week we are starting on our tech review which is gonna be a big project. We did not meet with our client because we did not have anything worth reviewing to go over with her yet. However we did meet as a group outside of class to go over what each of us was going to work on in the tech review. We started the process of writing but most of that will likely come down to this weekend and next week early on. I was assigned to talk about cross-platform development, sermon integration, and schedule integration.
			
			\paragraph{Kevin Stine}
			This week we got working on our tech review and I spent most of the week researching and figuring out what other technologies were available for us to potentially use for our application. I focused mainly on the Android development platform, Android UI design, and integration with a giving platform. Researching and seeing what other technologies are available was really helpful to get a better idea of what options we have moving forward. Moving forward I'll be starting to get our design document setup so we can start figuring out exactly how we want to design our app.
			
		\subsubsection{Week 8}
		
			\paragraph{Courtney Bonn}
			We finished and turned in our technology review this week. The next task is to work on our design document. I think we have a good feel for what we need to do, though we haven't begun working on it quite yet. My plan is to work on my portion of the design document at least a little bit before the Thanksgiving holiday. After that we will have the progress report along with the recorded presentation to work on for finals weeks.

			\paragraph{Max Dimm}
			I got a bit of a late start on writing my part of the tech review. I spent most of the weekend and Monday working on my pieces of it but was able to get my portion done in time. It took a bit longer then I expected as the amount of sections each part was fairly large. This week we will begin looking forward and running our tech review by our client. We also did not meet as we figured the whole meeting would just be reading through the 23 page doc. We decided just sending them the doc to read on their own would make more sense.
			
			\paragraph{Kevin Stine}
			This week we wrapped up our tech review and got it turned in. I spent a lot of time looking into the Android UI design that we want to use and it was good to dive deeper into the layout that we want to use in our app. Next we have the design document which will require more detail and figuring out exactly how we want to implement our application.
		
		\subsubsection{Week 9}
		
			\paragraph{Courtney Bonn}
			With Thanksgiving holiday this week, we didn't meet with our TA or work on the design document very much. I began the outline for the document and began working on the first four sections. This upcoming week will be very busy with the design document as well as hopefully meeting with the client to go over our decisions. We will also begin working on the progress report and planning how we will complete the recorded presentation.

			\paragraph{Max Dimm}
			We had a short week this week only meeting on Tuesday. We wrote to our TA Vee a short bit about what we were excited about and what we were worried about. Other then that we started our design document but I was not able to contribute much at this time due to being out of town. Next week I plan on getting my share of the work done whatever that may be. We also did not meet with our client due to the holiday weekend but we will most likely meet with them the following week.
			
			\paragraph{Kevin Stine}
			This week was a short one as we had Thanksgiving. I didn't really do much for this class this week but have begun thinking about the design document. I'll be doing a lot of research into the various viewpoints that we'll want to incorporate into our application. We'll probably meet with our client next week as it'll likely be the last time before Christmas break so we can answer any questions they might have.
			
		\subsubsection{Week 10}
		
			\paragraph{Courtney Bonn}
			With this last week of the term, we really drilled down and focused on the design document. I think we underestimated the depth of this document. Because the app that we're designing is supposed to be a simple design, we didn't realize how much effort and time would go into planning the design. We did end up finishing the design document and I think we did a great job. We also met with our client for the last time this term. We made sure we were all on the same page and answered any questions the client had. I have a good idea of where this project is going and I think it's going to be successful. During the break, I plan on doing a lot more research on iOS and Android development so that way when we start implementation in the Winter I know more of what I'm doing.

			\paragraph{Max Dimm}
			We got off to a bit of a slow start on the design document but were able to finish it on time. We were not super clear on the depth that was required by some of the sections/viewpoints. We were able to meet with our client before the end of the term which was good. We were able to send her the documents we had been working on and just make sure that our visions were still in sync for the project.
			
			\paragraph{Kevin Stine}
			This week we got going on the design document. Since we are all pretty new to mobile development, we weren't entirely sure how to go about mapping out our design processes for the entire application. As we begin to learn more about mobile development I'm sure we'll be able to add more in-depth information to our design document, but we had to get as much planned out based on our current skill level. We also met with our client for the last time before break so we could make sure to get on the same page about everything moving forward. Once break begins I plan on heavily researching iOS and Android development. I'd like to have a full-functioning layout of our application for both platforms before Winter term begins so we can focus on integrating the Church's database and information into our app without worrying about the layout.
		
	\subsection{Winter 2017}
	
		\subsubsection{Week 1}
		
			\paragraph{Courtney Bonn}
			During Winter break, I didn't accomplish nearly as much as I wished I had. I ended up being very busy with work and family obligations and wasn't able to work on this project. I have started looking in to Android Studio as I am using it for another class so this will be beneficial for our project. My goal within the next week is to finish up the Swift tutorial so we can get going on our project. Kevin set up the skeleton of our iOS app and we verified we were all able to pull it from Git and run it. I've also touched base with our client. We've agreed that we don't have any reason to meet thus far and will put our first meeting off a few weeks until we have something to present.

			\paragraph{Max Dimm}
			Over the break I really only downloaded x-code and experimented with it without much real purpose. I looked up a few guides but knowledge did not really stick. The first week I spent my time downloading getkraken on my MacBook and downloaded Kevin's skeleton. We did not meet with our client because we didn't really have much to go over with her but we plan to do so in the next week or two I think. My goals for next week are to start looking over some of the swift guides online and start learning the language.
			
			\paragraph{Kevin Stine}
			Over break I really didn't have much of an opportunity to dive deeper into App Development besides for just looking at some of the iOS documentation. I was able to setup a pretty basic app with multiple pages like we want for our application for Calvary Chapel. This will work pretty well as a foundation for building up the other aspects of the app, however there is a lot that we'll need to do in order to get this app fully functioning in the next few months. My next steps will be to dive deeper into looking at the documentation for the various APIs that we wish to use in order to bring in the functionality for the calendar or sermons.
			
		\subsubsection{Week 2}
		
			\paragraph{Courtney Bonn}
			This week I focused heavily on learning Swift and Xcode. I wasn't able to work on it during break, so I do feel a little behind in terms of our project. We have a basic template for our iOS app that has different pages and icon images, but our next step would be to start pulling in the actual information. I've emailed our client to get some clarifying information on the Church's new website. I believe the new website is using Wordpress (from looking at the site and the Web Inspector tool via Safari) which will allow us to use Wordpress' API to bring information to the app--allowing the web development team to only have to update the site and it will in turn update the app. I'm waiting for an answer on this now and will then start looking into Wordpress API further.

			\paragraph{Max Dimm}
			This week I began the process of looking over the swift guides we found online. I don't think I'm currently where I want to be with learning the language just yet. It's starting to feel like it might be one of those things where I just need to start working on it and look up the parts as I go. I believe my next steps are to either start working on the sermons functionality or the donations. I will need to see how to go about how to implement the IOS media player in the app so we can show the messages.
			
			\paragraph{Kevin Stine}
			This week I focused more on getting familiar with iOS development and I continued to refine the basic framework of an app that I created earlier. I've mostly been adding small features and getting a feel for how everything is laid out in Xcode so that as we move forward I'll have a better grasp on the platform we're using. I will begin working on getting the events page setup with a table view that pulls from the Church's database and populates the table with the events as well as the dates.

		
		\subsubsection{Week 3}
		
			\paragraph{Courtney Bonn}
			This week I worked on getting Calvary's logo on each page of our iOS App. It took a bit longer than I expected because even though it seems like an easy concept, you have to add constraints so that way it will be in the same position on every size iPhone. I received confirmation that our client is using Wordpress for their new website and that can have access to it's API. I've been busy doing a lot of research on using Wordpress API while waiting for the Wordpress access. Once we get access, we can use REST API and parse using JSON (this will obviously be separate from the XML parsing).

			\paragraph{Max Dimm}
			I started working on implementing the video player this week for the sermons. I have a pretty good idea of how it all needs to go down but I haven't started the implementation on our project just yet. I started by implementing it on a separate project to make sure I could do it before making edits to our actual project. I was able to get it to play a youtube video but need to make it work with livestream because that is the platform our client will be using.
			
			\paragraph{Kevin Stine}
			This week I got more familiar with Xcode and configuring the build settings for our application. I also began working on getting a table view setup so that we could get the Events (calendar) view setup to have the framework before we begin populating the table with values from the database. We are still waiting on our client to get access to an admin account which will let us connect to the API, but once that's done we should be able to the XML data parsed using NSXMLParser and pulling in the correct data for the events.

			
		\subsubsection{Week 4}
		
			\paragraph{Courtney Bonn}
			This week was a busy week in regards to our project. I worked on the bulletin page on the iOS app and was able to successfully get a JSON parser up and running. The parser is currently getting data from a JSON dummy site because we still don't have access to the Wordpress API. Once we get the Wordpress API credentials, I should be able to just change some variables and the URL in the current code to point to the Wordpress site and should be able to print the data from the bulletin page. I also began the Android app. I got a simple app up and running with side bar navigation drawer that goes to each page we need. This took a lot of time and was more difficult than I expected. The Android studio interface itself is not as intuitive and user friendly as Xcode. We also began work on our Winter Progress Report and sent a first draft to our TA to make sure we are on the right track. Up next we need to focus on the Android app and get that quite a bit further. We plan on meeting with our client on Tuesday 2/7 to have them look at what we have so far and get some questions answered.

			\paragraph{Max Dimm}
			Ok, so I came upon a stumbling block this week but was able to figure it out by the end of it. I was confused about linking the visual aspects of the code to the ViewControllers. Whenever I went to connect the two it seemed to link me to an uneditable file. By the end of the week in class I was able to figure it out with Courtney's help which was awesome. So now the same video player that I made for the separate app is now working with our live project. I just wanna make it work with the livestream player which shouldn't be that hard.
			
			\paragraph{Kevin Stine}
			This week I mainly stayed focused on the events page which I've been working on. I spent a lot of time getting the UITableViews setup so that the events could be displayed in a table form. I got stuck a bit stuck on creating a class for the UITableView, so I began working on the implementation of parsing the XML data from CCB. Once we were able to get the username/password for connecting to the CCB Public API, I was able to write a quick python program which connected to the database and then parsed through the XML data. I then got started on implementing a similar program in swift which is what I'm currently working on now. Once I get the XMLParser to work in Swift, I'll focus on getting the class to work so I can pass the data from the XMLParser into the UILabels.
			
		\subsubsection{Week 5}
		
			\paragraph{Courtney Bonn}
			This week I started working on our Android app. I was able to create a basic app with an empty activity. After some trouble, I was finally able to get a sidebar navigation to work so now our app has the different pages we will need. I also began researching how to parse JSON data and display it in Java/Android. This is a harder job than I expected and I've had a tough time finding a good tutorial that will help me with this. We don't need a complex parser for the bulletin page and most things I'm finding on the internet are a little more complex that what our project calls for. This next week I will continue researching to try and get this part of the Android app up and running. At our meeting this week we were able to show the client the products we have so far and they liked the direction we were going with the color scheme and the simplicity of the application. We still haven't gotten the API credentials for the Wordpress website so until then, we are stalled on the bulletin page. Our client asked us to look into how easy/possible it would be to offer the app in different languages and this is something we are currently researching. They said at the very least they want to be able to have the welcome message on the homepage in different languages. We said we could do this in a scroll fashion where the different messages scrolled on the screen through the different languages. The client agreed that if we weren't able to have different languages for the app altogether, we could have the message scrolling. This next week we have our progress report due. We haven't started working on it yet and I'm a little stressed out for time but I think we should still be able to finish it by the deadline.

			\paragraph{Max Dimm}
			I did some research about the live video embed link that livestream gives out but it seems that our client will need to upgrade their account to the premium service to access it. Because of this an a few other reasons we met with our client and I got some feedback on that. It seems they will either upgrade their account or upgrade their infrastructure and include youtube, either will work for me. So now I gotta start working on adding buttons to change the viewable video and start looking into the donations page.
			
			\paragraph{Kevin Stine}
			This week I continued my work on the Events page for iOS and got the basic framework for the events to work and was able to fix the issues I was running into previously where the tableView wasn't working properly with the eventView. I was finally able to get the events page to work without crashing by recreating the class for the UITableView. It took a while of checking online to try and troubleshoot the issue, but now that I have that part working, I can work more on the XMLParser and the details page for the events page.
			
		\subsubsection{Week 6}
		
			\paragraph{Courtney Bonn}
			This week we focused on writing our progress report and recording our presentation. Our apps are not exactly where we wanted them to be for the alpha release so we spent some time getting them a little further before we finished our presentation. I worked more on the Android app and added a home page fragment and then added that page to the navigation. My next step is to get a JSON parser working on the Android app. We are still waiting for Wordpress credentials so I'm still not able to pull the correct data to the bulletin page.

			\paragraph{Max Dimm}
			I was able to put aside most of my other work to put time into the presentation this week. I was able to write about 800 words on my experience and record voice over for the first 7 or so slides. I pushed any changes I had made to the app so that when Courtney and Kevin did their recordings of the app demonstration that they would see any changes that I had made up through this week. I need to work on implementing a few bits of functionality to the app that will increase usability.
			
			\paragraph{Kevin Stine}
			This week we focused mainly on our progress report and the recording for our presentation. As I was able to get the Events table view page to work, I mostly cleaned up a few things on my code which allowed me to display test data for the various labels that I have for the event name, day and month. Next I will be working on getting the XMLParser working since now that I can display data, I just need to parse the correct data from the CCB API.
			
		\subsubsection{Week 7}
		
			\paragraph{Courtney Bonn}
			This week I focused on getting a JSON parser up and running on the Android app. It was frustrating at first, but after a few hours I was able to successfully get it working. At this point, it's still pulling placeholder data because I still don't have the Wordpress credentials I need in order to access the API. On the iOS app, I changed the JSON data that was printing to the screen to see if I was able to grab other JSON data. I didn't work on the project nearly as much as I wanted to this week due to being busy with other schoolwork. Next week I'm hoping I get the Wordpress information I need and then I can make some headway on that.

			\paragraph{Max Dimm}
			I worked on adding buttons to change the video for past sermons as opposed to leaving the video player constantly on the live feed from livestream. I think I have a method for doing what I wanna do ready, but just haven't implemented it yet. I expect by the end of this next week I will have it done and be working on implementing the donations page for the IOS app. We met with Vee and gave him a status update, he said we were perhaps a little behind schedule but not so badly that we need to panic. We just need to keep steady forward progress at this point.
			
			\paragraph{Kevin Stine}
			This week I mainly focused on continuing my development on the XMLParser section of the events page. I was able to get the basic framework completed that would parse through some sample XML data and display the names of food in the label. It took a while to get this basic functionality up, but after trying to connect with the CCB API through making an URL call, I was getting error messages saying that the data stream could not be read. It looks like I've got some debugging to do to get the URL connection to work properly.
			
		\subsubsection{Week 8}
		
			\paragraph{Courtney Bonn}
			This week I finally received the information needed for the Wordpress API. I was able to enable the JSON API on WP and switch out the URL I was pulling JSON data from for the correct bulletin page. It took a little bit of finessing to figure out how to display the correct data from the JSON data though. Eventually, I was able to pull just the content and display it on the iOS app. However, I figured out that the CSS styling doesn't come over and the data displayed all of the HTML tags. After some further research, I found that if you display the data in a UIWebView, you can keep the CSS styling. So I was able to still parse the JSON data natively, rather than just loading the whole page in a web view, but I was also able to load the content I needed in a Web View so that way I can style it correctly. Currently, I still need to work on the CSS styling a little. It doesn't look exactly how it does on the church's website so I do want to put more work into and make it more consistent. My next step is also to work on the Android portion and get the bulletin page displayed on that application as well.

			\paragraph{Max Dimm}
			I figured out and implemented the buttons that I wanted to last week. I just need to repeat the process a few times for each of the buttons but thats just a matter of doing it. I'm now researching a method for implementing the donations page. I think I have exactly how I wanna do it nailed down, I just gotta give it a try. I wanna just display a html view of the existing donations page, as thats what our client thought would be best. That way we don't have to mess with their private banking info and risk having a security dilemma be on our hands.
			
			\paragraph{Kevin Stine}
			This week I continued to work on the XMLParser for the events page. As I was getting an error message saying that the data stream could not be read from the XMLParser, I did a lot of digging online and found relatively few results. I tried doing everything that the posts online mentioned however I still was not able to connect to the URL. I tried connecting with a test URL and was able to connect with no issues. I was able to narrow down the issue to the way that our username/password is setup for accessing the API. The username we're using is bonnc@oregonstate.edu and the URL connection call I was making was: https://bonnc@oregonstate.edu:password@url-to-CCB.com I was able to determine that the @ is probably what's causing the issues, since when I try doing the same thing using CURL, I'm given some information from an Oregon State page. We let the client know that we need to change the username and we should be able to connect without issues after that.
		
		\subsubsection{Week 9}
		
			\paragraph{Courtney Bonn}
			This week I made a lot of progress on the Android app. I finished the bulletin page after receiving the correct information from the Wordpress site. I was able to have the raw HTML code displayed on the app. Obviously, this isn't exactly what we're wanting to display. After some research, I discovered that I needed to load the JSON data into a WebView and then the CSS styling would be kept in tact as well as the necessary html tags (<\\p>, <\\strong>, links, etc.). Once I correctly loaded the data into a WebView, the data then printed correctly onto the app. The links are currently being printed in a bold blue, which at some point I would like to change. But that is just polishing the functionality of this page is completed.

After I finished the bulletin page, I decided to begin working on the donations page. Our plan for this page was to just load the entire page into a WebView so that way our app wasn't dealing with any of the security factors such as storing credit card information. While this was relatively easy, I figured out by loading the whole page, it would also load the menu, the header, and the footer of the Giving page. Since we already have a header on our app, we definitely don't want two. I went back to researching, and discover if I injected javascript into the code, I'd be able to remove certain elements of the HTML. I used this to remove the header (which removed the menu) and the footer and now just the content loads.

Right now I have a few things I'm struggling with that I haven't found the answer for. 1) The bulletin page loads very slowly. The header image and the title of the page loads immediately, but there's a slight delay while the JSON is parsing before the actual content is loaded. The functionality is correct though, so this will be something I focus on a little later. 2) When removing the header and footer from the donation page, the entire donation page loads for a split second and then reloads without the header and footer. In other words, the user can see the header for just a second while it reloads without it. I am not sure how to handle this but again, the functionality is there so this isn't something I'm super concerned about. 3) I'm worried about finishing our Android app. The iOS app is much further along. Currently, only the bulletin and donation page is finished--the Events page and the Message page aren't done yet. Because we're only a couple weeks away from the end of the term, I'm nervous that we won't be able to finish this app.

			\paragraph{Max Dimm}
			After seeing how calvary updated their website, I decided I didn't like my implementation of the sermons page and went about changing how the user would view past sermons. I went with creating a link that would open the users default browser over our app that displays Calvary's most recent videos and content on livestream. The reason I went with this is because it was near impossible to keep the links for past weeks updated without updating and relaunching the app every week. This wouldn't work because the app would need to be updated on the weekly in order to display the correct videos for the past weeks.

I need to start making moves towards implementing the sermons page on the android app. I haven't been contributing in that area at all which is not good. Worst case scenario I would like to have my IOS portion done or basically done by spring break so that I can put that behind me and work on the android app full time spring term.
			
			\paragraph{Kevin Stine}
			After finding out the issue with the username for our XML parser, we contacted our client to see if she could update our username. It took a few days to get the username updated, and when I tried connecting I was getting a different error message saying invalid credentials. It looks like the client may have updated the username, however they didn't update the admin once which prevented me from connecting. Once we get the new updated admin username I should be able to finish up the iOS events page.

			
		\subsubsection{Week 10}
		
			\paragraph{Courtney Bonn}
			Because of other final projects due before dead week, I wasn't able to work at all on the capstone project. Where the app stands now is a 75\% finished iOS app and a 50\% finished Android app. The iOS app has almost every page at full functionality, though it will still need work aesthetically. The Android app has two pages at full functionality--the bulletin page and the donations page. The events and the messages still haven't been implemented at this point. The welcome/home page on both apps are set up but could still use some personalization from the church. Because this isn't a matter of functioning, this will be part of the "polishing" stage next term. In the next week I will be working on my final report for the term, as well as the presentation as a group. Before the end of this term or early next term, the goal is to be functionally done with both apps and then we can focus on the aesthetics's and user testing.

			\paragraph{Max Dimm}
			I fixed the sermons page to be as I want it to be. All it needs is a visual touch-up. Next I'm looking at the donations page. I have implemented it similar to how it is done on the android app but need to figure out how to strip the header and footer on swift like Courtney did. I was having trouble finding resources that shed any light on if that is even possible. However, at a very minimal level, the donations page is functional. It just has more options or into on the page then I'd like. This week has been busy as its leading up and into finals week as I have projects to finish and finals to study for so time has been in a crunch.
			
			\paragraph{Kevin Stine}
			With the discovery of the issue on the credentials we were using, I was able to finally fix the issue once we got our new username. With that complete I was able to move over my code from a test app that I was using into the master branch. Since the app I was testing on was built slightly different than our main app, there are a few bugs that will have to be squashed before I can get the events page fully functioning. Once the bugs are taken care of the events page should be almost 100\% complete.
			
	\subsection{Spring 2017}
	
		\subsubsection{Week 1}
		
			\paragraph{Courtney Bonn}
			This week is our first week back after Spring break. I didn't work on the project during Spring break. This week we met with Vee and gave him the summary of where we are in the project. Because my pages are functionally complete, I'm now focusing on researching how to improve my pages. The first thing I started to research was how to strip the header and footer from the UIWebView donation page in order to help Max out. Like him, I was struggling to find a solution. The closest I could find was embedding javascript in an enablejavascript function using the WKWebView. I gave the info to Max to continue researching. Next I moved on to my bulletin page on the iOS device. I need to deal with links within the bulletin. I've researched adding a back button and also having the links open in safari. So far I haven't made a decision, but I'm going to continue researching and will then begin implementing the solution I find. Once I decide on a solution, I will move to the Android app and deal with links at that app.

			\paragraph{Max Dimm}
			Being the first week of spring term I know I got to get going on the android portion of the app. I was having trouble doing git pulls on the command line so I went and downloaded gitkraken which has helped a bunch. I got the app up and running on android studio and stopped for the most part at this point. I began doing research into how I wanted to implement the sermons page (similar to the apple app) and wanted to see if a similar method would be possible on android studio. I have not done much coding in java before so this is slightly uncharted territory for me but I think it is entirely doable.
			
			\paragraph{Kevin Stine}
			Continued working on the events page now that we have the proper username which is actually working so I'm able to connect to the database and parse the XML. I was able to get the correct data parsed from the XML however I ran into some issues when trying to pass the variables that got saved from the XML to the actual UI labels in the app. It looks like the data gets parsed correctly however I'll need to do more research on getting the labels updated.
			
		\subsubsection{Week 2}
		
			\paragraph{Courtney Bonn}
			This week I made progress both on the iOS and Android app. On the iOS app I added a function that will force links on the Bulletin page to open in Safari, rather than in the app itself. I also helped Max out on the donation page and figured out how to remove the header and the footer. I've already done this on the Android page and the code was very similar, but I had to work out how to convert it to Swift. Also on the bulletin page I was able to update the CSS to match (as closely as I could) the stylesheet on the corresponding website page. The font is slightly different because Calvary is using a custom font, but it's very close. I also fixed the fonts/color of the headers on each page to match the website. Additionally, I helped Kevin out on the Events page. He wasn't able to get the events printing on the app, just in the console so I debugged and worked through the code until I was able to get it working. The event name and the date are now printing to the app in a Table View.

On the Android app I got the base files ready for the Events page, which Kevin will take from there. I found a good resource for an XML Parser and got the layout coded as well. This next week I am waiting to hear from our client on a question about their LiveStream account. According to my research, we may be able to access the LiveStream API and with our client's secret key we could pull the most recent video (past event) to display on the app. Currently the video playing is hard coded. I've contacted the client and she is checking with the web developer.

			\paragraph{Max Dimm}
			This week I don't have a ton to report on because I did not get a ton done. I was able to find a guide online that was exactly what I was looking for. I had a lot of school work I had put off and ended up not spending a ton of time on the app. I began working on implementing the steps I found in the guide but did not finish. The app was not building at the time which was concerning but I was confident in my ability to get it to work without too much struggle.
			
			\paragraph{Kevin Stine}
			This week Courtney was able to help me out with the events page and she was able to get the events to display on the events page and parse through the different dates. Now that we're able to pull in one specific week, we need to update the way that we parse the URL so we can accept more dates than just that week. This will require a bit more research on the CCB API to figure out how to specify an end date for the API and also figure out how to specify that end date based on the current days date.
		
		\subsubsection{Week 3}
		
			\paragraph{Courtney Bonn}
			This week began cleaning up some of the code for my bulletin page. This included fixing the CSS and making it match the current website. I added the CSS code to both the iOS and the Android page, meaning these pages are completely finished on both apps. I also fixed the icons that were in the navigation drawer on the android page--they are now matching the icons we have in the iOS app. I then removed unnecessary files/xml layouts that were in the Android app but we weren't using. I set up a meeting with our client for April 27 to discuss the project and where it's currently at. I asked our client to see if we could get a LiveStream API key, but they're having issues with this so we may or may not be able to complete this.

In the next week I am going to help work on the XML Parser on Android to get that finished. Because of the May 1 deadline, I'm going to do my best to work a lot on the project to get it completed and ready to test. Testing will more than likely take place after the May 1 deadline but before expo.

			\paragraph{Max Dimm}
			This week I was able to get the sermons page up and working. The functionality is there but I'm currently having issues with getting the video to fit in the window I have created for it. I spent a lot of time working on resizing the video and the window but no matter what I do it seems there will always be a scrolling option which I would like to avoid. From this point I am looking to fix that issue and have a button implemented that will give users the ability to watch past sermons. From there I can start helping fix other small issues the app may be having.
			
			\paragraph{Kevin Stine}
			This week I was able to implement the logic which allowed me to set an end date to the url based on the current days date. I'm now able to display more than just the current weeks events but up to a month which is as far out as we thought would be necessary. From here I'll just need to figure out the events details page and pass the data for that particular event to the next viewController using a segue.
			
		\subsubsection{Week 4}
		
			\paragraph{Courtney Bonn}
			This week we worked on trying to get all of the little details finished before the code freeze on May 1. Because we were still having issues finishing some of the bigger pieces, I did feel a little behind before this week started. This week I focused on the Events page on the Android app. Originally this was assigned to Kevin, but he was debugging and having issues with this page on the iOS app and since I had already finished my pages, I took on the android page. I implemented an XML Parser and was able to successfully pull over the events. Now when you click on an event, you can get more information about that specific event. I also made it so it would automatically pull one month of events from the current date. I also discovered that the back button on the Android app didn't work properly -- it would actually exist the app. After some research, I determined that I needed to add the fragments to the back stack so that way when the back button was pressed, the fragment would know where to go. Once that was working properly, the back button no longer exited the app. At this point the android app was mostly complete except for a few little things. I was able to move on to working on the details. I was finally able to fix the constraints on the messages page on the iOS app. I ended up needing to delete almost everything on that layout and reading the components in order for the logo to show up in the correct position. I also added activity indicators on the bulletin and donation page (iOS app only) that will show while the page is loading.

After meeting with our client on Tuesday, we discussed some additional features that the church would like if we are able to add them as well as some changes to the existing interface. The first one was removing the extra sections on the donation page. Currently the entire history on where the money is going was on the page and our client decided that they just want the ability to donate and users can visit the website for more information. The second was adding a date selection to the events page -- giving the users the ability to change the month of events. I was able to complete this on the Android app. A user can now press "previous" or "next" to change the month of dates. There is no selection for the specific day because the client just wanted them to change the month.

The other requests our client had were seeing about the possibility of push notifications and adding forms for event registration. As these were not on our original list of requirements, we decided that we could look into this after the code freeze but focus on the requirements until then.

This upcoming week (at least before May 1) I have a few goals in mind:

	\begin{enumerate}
\item Try to get the bulletin page to load faster. I am not sure if this is possible. I've been trying to figure out if I can "call" the bulletin page from the home page so that way when the app starts up it automatically calls the bulletin page and starts loading it but so far I have been able to implement it. This could just be dependent on the users internet as I don't believe the JSON can parse/display things any faster.
\item Lock the orientation on both apps. The user shouldn't be able to go into landscape mode. Neither app was designed for this and when it does go into landscape, certain features don't work right.
\item Assist Kevin with the event details on the iOS app. It's currently crashing the app and I have some time to help so I'll see if I'm able to figure out why it isn't working.
\item Figure out why the back button isn't working on the event details in Android. Currently when you press back it doesn't return you to the calendar you were viewing before, it reloads the page and goes to the current date.
	\end{enumerate}

			\paragraph{Max Dimm}
			Ok, this was a big week and a lot was done on my end and in general on the app (well duh its the week of the code freeze). I got the sermons page fixed so that the scrolling issue is no longer a thing which was a big relief. I was able to implement the button to watch the past sermons fairly smoothly which was very pleasing as it works just as I wanted it to. Now its on to keep working on bug fixes wherever we see them existing. I've fixed the scrolling on the donations page (similar process to fixing the video scrolling) and added readme pages to both the android and iOS folders. Other QOL things i'd like to fix are adding loading wheels to some of the pages if possible and helping make sure the app stays locked upright.

We met with our client for the first time in a long time and we got some good feedback on the app. However they wanted to add a few functions to the app which we said we would give an attempt but couldn't make any promises. This was due to getting the word on them 3 days before the code freeze. Had they let us know about this functionality at least a few weeks earlier it would have been more feasible.
			
			\paragraph{Kevin Stine}
			With the events page now working for up to a month out, we focused on getting the details page working. Courtney helped debug some of the issues I was having where the variables would not actually be passed through the segue to the next view controller. She was able to get that implemented so now the details page displays everything from start time to leader phone number.

This week we also met with our client for the first time in a while and got some feedback. They wanted to add a few more features which we said we would look into but for the sake of getting everything done for the code freeze we would mainly be focusing on the parts of the app that we specified in the requirements document. One feature that they wanted implemented was a date picker which would allow the user to specify a month or a year that way they would be able to look into the future (or past if they wanted) and view all of the events. Initially this got setup using a pickerView, however it was a little clunky and required us to hardcode the years. I was able to switch this over to a DatePicker which now allows the user to select the month and year in which they would like to view. From here until the code freeze it's just fixing bugs and making everything is functioning properly and looks good.
			
		\subsubsection{Week 5}
		
			\paragraph{Courtney Bonn}
			This week I worked on getting the event details printing on the iOS app. Once I got some details printing on the new view controller, I was able to parse additional XML information from the event and print those details as well. I also added back and forward arrow pictures on the Android events page. I then worked on implementing date selection on the iOS events page. After I finished, we decided to go a different, simpler route and Kevin ended up redoing the date selection. After that the bulk of the code was finished and we just had a few finishing touches before the code freeze. I updated the welcome message per the client and removed unnecessary code. On the bulletin page on the iOS app, I added a pull to refresh function as well as removed a grey background that showed when pulling the web view down. I also received a new photo to put on the home page from the client.

			\paragraph{Max Dimm}
			This week I did a few usability changes. Small things like changing the messages button on the IOS app and removing a text field that was not intended to be there. I changed how the button looked on IOS so that the users would be more able to identify it as a button. Outside of this there was some research I did into push notifications but I did not delve too deep into that. I went and added readme's to both the android and IOS folders so that anyone looking at our code would know how to run and compile our apps.
			
			\paragraph{Kevin Stine}
			This week I did some more work on the Events page. Courtney had implemented a way for the user to pick a date, however it required our years to be hardcoded which would require the future developers to go in and manually update the app. I began working on a different implementation which utilized the built in DatePicker with a toolbar for selecting the month and year. While the date picker also includes a specific day, our client didn't want the users to be able to specify the day. So now when you select a month and year, it will default to automatically loading that entire month's dates, rather than starting with the particular date the user specified. Since this was an addition that our client wanted to add a week prior, we implemented it as best we could and if they do want to allow for specific date selections in the future it'll be pretty easy to implement. I also went in and cleaned up a lot of the Events page that had debugging statements or old code that we were no longer utilizing.
			
		\subsubsection{Week 6}
		
			\paragraph{Courtney Bonn}
			Now that development is essentially finished, I took a step back from this project this week to focus on other classes that I had put on the back burner. The next step for this project was to begin work on the midterm progress report and presentation. I got our report set up and began working on the overview sections for Fall and Winter terms as well as the future work section. In the future work section I described the additional content the client may add after we had the project over. In the next week I will work on getting the progress report finished and focus on preparing for expo.

			\paragraph{Max Dimm}
			Truth be told, not a ton of focus was given to capstone this week. I figured we had finished development so It was bumped down a notch in my priority lists. We began working on the midterm progress report but I think I will work on it further next week as I am heading back to Portland for the weekend. I need to handle the writeup for my pages and record voice for my part of the slides but that should not take too long.
			
			\paragraph{Kevin Stine}
			With all our major development completed (everything on our requirements document), I used this week to focus on other classes that I had put off since we had a big push to finish everything for this class. Next is getting our midterm progress report and presentation completed which I'll probably work on this coming weekend. Other than that just preparing for expo is all that I have for this week.
			
		\subsubsection{Week 7}
		
			\paragraph{Courtney Bonn}
			Now that we have basically finished our project and expo is now over, we are at the point where we are getting ready to transition the ownership of our project to our client. Our client has indicated they are going to tweak and continue developing the apps for a little while before publishing them.

If I were to redo the project starting Fall term, I would tell myself to spend more time researching mobile development in the early months, rather than waiting until Winter term. This would have allowed me to begin developing more confidently earlier on and could have given us more time to add on features or do other cools things with the apps that we ended up not having time for.

I'd say the biggest skill I'm taking from this project is how to work with a team and a client over a long period of time. Before this class, the longest I'd have to work with people is a few weeks--a term at most and then I'd be done. Learning to cooperate and help each other while dealing with requirements and requests and meetings with the client was a great way to prepare myself for future projects I may work on at a job.

My favorite part about the project was learning how to develop apps in general. This has always been my main interest and now I have the actual experience to go along with it. The only part of the project I would say I didn't like is not being able to actually publish the apps. Since the client wants to continue working on them, we can't truly say the project was "complete" since they aren't published yet.

I learned a lot from my teammates but the biggest thing was just how to work in a team. Like I said earlier, I wasn't used to having to work with the same people for such a long period of time. But in this type of project, you have to learn to work with each other and find a good balance. I think we found a great balance between us. We each had our strengths and weaknesses and we could use those to help each other out.

I think I would be satisfied if I was the client. While there is an issue with one of the pages still, that problem has been adequately explained to the client and until they upgrade their system, there isn't anything we can do about it. Because we know about this issue and know how to fix it (we just can't control that part), I am satisfied with the project as is.

If the project were continued, I think the messages page would need worked on more. Once their system is upgrade, I think it'd be great to pull the last 5 or so events that were on the livestream page and have them all available on the app, rather than just the most recent. I also think adding notifications and the ability to register for events within the app would be cool features to work on.

Overall, I'm really pleased with our product and I'm proud at how much we were able to accomplish.

			\paragraph{Max Dimm}
			I am so glad we are done with expo, we just need to stay focused for a few more weeks and we're done. Looking back on this project, it was a lot of work, very frustrating, and I feel like I learned a ton. If I had the ability to go back and give myself advice back in fall term I would tell myself to just download Xcode and android studio and spend 2 hours a week just trying to implement a few things. I think this in and of itself would have taught me a ton and would have made the process of development much more smooth winter term.

In regards to the biggest skill i've learned, I would say it is version control. I had not used git (as it was intended) before. Seeing how it can actually save us when we make mistakes is amazing. Also, being able to coordinate code between three people without having to re-do things or overwrite code is awesome. This is definitely something that will come in handy down the road. Also, I feel like knowing the foundations of swift and java will be super helpful down the road and could easily be applied to future jobs.

In regards to my enjoyment of the project, I liked being able to see the results of my code so quickly and to port it onto my phone. This gave the project a very real feel to it as opposed to other projects I've done. The IOS app did not have the same feel to me because the simulator on my laptop was so poor. I enjoyed being able to show my friends the app we were working on and being able to show them the differences over time. In regards to things I did not enjoy about this project, I did not enjoy not having many solid resources to use. At times it felt like if the internet did not yield the answers I needed, that I was screwed because I didn't know who to ask on campus.

I learned a lot from my teammates throughout this. I think the biggest thing though was communication. We used slack to stay in contact with one another and I felt, at least, that our communication was fairly spot on. We let each other know what we were working on and were able to answer one anothers questions fairly easily. I think our group worked very well together and we ended up with a solid product.

I think I would be. Not because the end result was what we ended up wanting, but because the end result was what we asked for initially. Its hard to judge a group based on results when the desired result changed so much throughout the project and so late in the projects development life. I would have aimed to communicate better with my group if I was the client in this particular project.

I think moving forward this project may revive more focus. I know our client was interested in a few things like adding push notifications and geofencing. I think the client will need to upgrade their systems on their end to get live functionality added, but will eventually want someone to change that line of code for them. I think the design could use perhaps a tad bit of polish. I think the app in itself works great, but a few things added to it to give it more functionality would be nice.
			
			\paragraph{Kevin Stine}
			With Expo done it feels like a pretty big weight has been lifted off of my shoulders. Now that our app is fully functioning and working properly as per our requirements, we?ve been able to relax a bit more and take it a little easier. While our client still has a few extra things that they would like to see implemented that they requested a few weeks ago, the main functionality is done so at this point anything else we add would be during our free time. This week there wasn?t much to do besides Expo, which I thought went really well and it was fun to see everyone?s projects after a long year.

If I were to redo the project from fall term, I would probably have started learning Swift and Java sooner, probably during the middle of fall term. While we had a lot of documents and things to write, I think having a better understanding of how Swift worked, getting familiar with the newer version (3.0 vs older iterations) would have been helpful. It seemed like once I actually got to developing for iOS, a lot of the issues I ran into were due to following guides that were based on an older iteration of Swift. I think reading through the documentation earlier and really reading into the different functions that we would be trying to implement would have been beneficial. Once we got into Winter term I spent a lot of time researching the Date class in Swift and utilizing that versus NSDate, Calendar or NSCalendar. It seems like if I had focused on one particular implementation it would have made development a lot smoother.

I think the biggest skill I?ve learned would be problem solving. We were essentially given a problem with no guidelines besides what they wanted to accomplish. From there it was up to us to determine exactly what it was that the client was requesting, turn that into requirements and develop an application that would fit those requirements. Essentially the entire process was one big long problem solving exercise. I think it really helped develop my long-term problem solving skills. Up to this point most projects I?ve worked on had been at the most, a two week ordeal. However when you think in the context of months rather than weeks, things start to get more complicated and I think I was really able to hone my long term planning skills. I see utilizing those same problem solving skills in the future as I apply them in the workplace.

I really liked being able to develop a mobile application as it has always been something I?ve been interested in. I think it was really cool to take a concept and turn that into an actual product which could potentially be on the marketplace. I think for me, iOS development was a lot more interesting since it seemed like Xcode was a lot more polished than Android Studio. I think the portion that I liked least about this project was debugging and trying to figure out issues and bugs that I wasn?t able to find an easy workaround or solution online for. It took a long time for me to get the events page working due to a lot of those types of issues.

I learned how to communicate and work well with a team. Through using Slack we were able to keep everyone updated and communicate freely throughout the three terms. I think this really helped us as we didn?t have to meet in person too often (since that would have been inconvenient as we live in three different cities), and didn?t have crazy long email threads that got convoluted and messy.

If I were the client I would be satisfied with the work that has been done on the project. Based on the initial requirements and feature requests, everything is implemented at a functional level. While some of the new features and requests aren?t on there since it was requested last minute, I think that apps do everything that the requirements specified and in a manner which would be easy for the congregation to utilize.

While I don?t necessarily think this project has enough potential ideas to make it a good candidate for continuation, I think it could be a bit more polished with some of the newer features that the client requested. Things like notifications (which require a developer account) and other languages would be possible to implement, but I think all of the main functionality for the app is there and at this state, a pretty good app.
