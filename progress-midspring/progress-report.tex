\documentclass[letterpaper,10pt,draftclsnofoot,onecolumn,titlepage]{IEEEtran}

\usepackage{graphicx}
\usepackage{amssymb}
\usepackage{amsmath}
\usepackage{amsthm}
\usepackage{alltt}
\usepackage{float}
\usepackage{color}
\usepackage{url}
\usepackage{enumitem}
\usepackage{pstricks, pst-node}
\usepackage{geometry}
\usepackage{array}


\geometry{margin = .75in}

\usepackage{hyperref}



\newcommand*{\signature}[1]{%
	\par\noindent\makebox[3.5in]{\hrulefill} \hfill\makebox[3.0in]{\hrulefill}%
	\par\noindent\makebox[3.5in][l]{#1}	    \hfill\makebox[3.0in][l]{Date}%
}%

\def\name{Kevin Stine, Courtney Bonn, Maxwell Dimm}
\def\team{Calvary Chapel Corvallis}
\def\grp{Group \#62}

\hypersetup{
	colorlinks = true,
	urlcolor = black,
	linkcolor = black,
	pdfauthor = {\name},
	pdftitle = {CS463 Progress Report},
	pdfsubject = {CS463 Progress Report},
	pdfpagemode = UseNone
}

\begin{document}
	\title{\huge \team \\ Progress Report\\ CS 463 Spring 2017}
	\author{\large \name \\ \grp}



	\maketitle

		\begin{abstract}The purpose of this project is to produce an iOS/Android application for Calvary Chapel of Corvallis that will allow members to access a plethora of information all in one localized space.
		The Church's current website does not provide an interface where current members of the church can very quickly access important information such as events, bulletins, and messages from the service.
		The desired application will be simple enough for anyone to use while providing back end access for staff to easily upload new information to the app.
		The priorities lie in maximizing the usability of the app and providing bulletin, schedule, video, and giving functionality.
		We will work with the existing Calvary Chapel web development team to create a product that is seamlessly integrated with their already existing network.
		\end{abstract}

		\clearpage

\section{Purpose and Goals}
		The purpose of our project is to create an application for Calvary Corvallis Church that will act as a connection between the congregation and the administration.
		The church already has a website that has some of this information, but they want the website and app to serve different functions.
		The website will be to introduce people to the church.
		The app will be used for the existing congregation as the go to place to access the most commonly used or needed information.
		Some of the features being provided within the app are: having sermons available, listing the bulletin, having the church schedule, and allowing members to donate to the church.
		Our client explained to us that these were the features that they wanted in the app as they are the most needed services by their members.

		Our client has requested that the app be as automated as possible in regards to updating the information hosted within it as to reduce any upkeep as much as possible.
		So we will be working with their existing infrastructure as much as possible to pull our information from.
		We are also creating both an iOS and Android application and we want to make the applications as functionally similar as possible.
		This will allow for greater understanding of the app across users who may or may not be super tech savvy.
		Our final goal in this project is to reduce costs wherever possible for our client.
		If that means suggesting newer cheaper infrastructure or setting up our app in a way that reduces how often it will need to be updated, we want to do it.

\section{Fall 2016}
		During the first few months of working with this project, we focused on learning what our client wanted, determining the exact requirements, designing the projects, and learning how to develop mobile applications.
		 Though no implementation took place during this time, a large bulk of the design decisions and specifics about the applications were completed which set us up for a smooth transition into development. 
		The requirements of the project were decided between us and the client. 
		The main requirements included a bulletin page, a calendar, the ability to donate, and the ability to watch the most recent sermon. 
		Additionally, the client wanted a simple interface that looked similar to their current website and was not cluttered with too much information. 
		Other requirements included having the application available on both iOS and Android smart phones and working with their existing back end software and website. 
		
		To meet the requirement of having the app available on multiple platforms, we decided to create two native applications using Xcode and Android Studio. 
		We used the incorporated Interface Builder to design the iOS app and we used a navigation drawer system in Android Studio. 
		The bulletin page was originally going to work with their back end software, Church Community Builder, but after the church changed their current website to Wordpress, we were able to pull directly from their website page. 
		The events are being extracted from Church Community Builder and the messages page is working with LiveStream. 
	
		
\section{Winter 2017}
		Most of the implementation of the iOS implementation was done during Winter term. 
		The first couple weeks were spent continuing to learn how Xcode and Android Studio worked and getting our base applications up and running. 
		
		
\section{Current Status}

\section{Problems Encountered}

\section{Future Work}
		Within the last few weeks, the client has realized there are additional features they would like to be present on the applications. 
		Because of time constraints, we were not able to complete these features for the client. 
		However, the client does have a web development team that we will be turning the project over to and they will continue to develop and add to what we have developed. 
		
		The additional features include push notifications, multiple languages, the ability to fill out registration forms for events, and possible additional pages. 
		
		For the push notifications, the client wants to be able to alert their members when there is an event cancelled, such as a church service, or other emergency alerts. 
		At this time, we are unable to implement push notifications. 
		On the iOS app, notifications require a developer account to be able to implement and test. 
		Our client does have a developer account and will have the ability to add notifications. 
		On the Android app, notifications are sent out through Google Cloud Messaging which would add an additional service the church would need to keep track of. 
		Our client has not decided whether or not they would want to add this service, so notifications will not be implemented on the Android app right now. 
		
		About halfway through development, the client wondered about the possibility of making the app available in different languages. 
		Because this was not part of the original requirements, we set it as a stretch goal to complete if we had finished all other requirements and still had time to spare. 
		As of this time, we were not able to implement this for the client as the original requirements took longer to complete. 
		
		Recently, the client wanted to add the ability to fill out registration forms for particular events. 
		The theory would be that a person could click on an event, see a registration link, and fill out the form directly on the app and submit it to the church. 
		Currently, we are trying to test this function with test events and are waiting for the client to add a test form. 
		More than likely we will not be able to finish this implementation before we are finished with the project. 
		
		The client's senior staff has ideas of future pages or content that could be added to the app. 
		Adding pages to the apps is a relatively easy task and will be explained to the development team when we are finished with the project. 
		They are not sure what kind of content may be added, but they want to keep the possibility of more pages open.
		
	




	

\end{document}
